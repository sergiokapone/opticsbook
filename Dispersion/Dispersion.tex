% !TeX program = lualatex
% !TeX encoding = utf8
% !TeX spellcheck = uk_UA
% !TeX root =../OpticsProblems.tex

%=========================================================
\Opensolutionfile{answer}[\currfilebase/\currfilebase-Answers]
\Writetofile{answer}{\protect\section*{\nameref*{\currfilebase}}}
\chapter{Дисперсія}\label{\currfilebase}
\makeatletter
\def\input@path{{\currfilebase/}}
\makeatother
%=========================================================


%%% --------------------------------------------------------
\section{Основні поняття і закони}
%%% --------------------------------------------------------

Дисперсія світла --- явище, обумовлене залежністю показника
заломлення речовини від частоти електромагнітних хвиль (або довжини
хвилі), які в ній розповсюджуються:
\begin{equation}\label{}
	n = f(\lambda).
\end{equation}


Якщо показник зменшується зі зростанням довжини хвилі $\frac{dn}{d\lambda} < 0$,  то дисперсія називається \emphz{нормальною},  якщо збільшується --- $\frac{dn}{d\lambda} > 0$, то \emphz{аномальною}. Аномальна дисперсія має місце на вузьких неробочих
ділянках спектру поблизу ліній поглинання.

Конкретні оптичні матеріали характеризуються середньою
дисперсією показника заломлення речовини:
\begin{equation}\label{}
	\frac{\Delta n}{\Delta\lambda} = \frac{n_{F'} - n_{C'}}{\lambda_{F'} - \lambda_{C'}},
\end{equation}
де $n_{F'} - n_{C'}$ --- середня дисперсія, яка визначається як різниця показників заломлення для блакитної (F' ) і червоної (C' ) ліній кадмію.


Кутова дисперсія призми визначається при $n_0 = 1$:
\begin{equation}\label{eq:Prism_Dispersion}
	\frac{d\sigma_0}{d\lambda} = \frac{2\sin\frac\theta2}{\sqrt{1 - n^2\sin^2\frac\theta2}}\frac{dn}{d\lambda},
\end{equation}
де $\sigma_0$ мінімальний кут відхилення променя призмою; $\theta$ --- заломлюючий
кут призми; $n$ --- показник заломлення матеріалу призми.

%---------------------------------------------------------
\begin{figure}[h!]\centering
	\begin{subfigure}{0.45\linewidth}
		    % PRISM + REFRACTION
\begin{tikzpicture}[scale=2]
    \def\N{200} % number of rainbow rays
    \def\L{2.8}
    \def\na{1.0} % air
    \def\angi{50}
    \coordinate (O) at (0,0);
    \coordinate (R) at (\L,0);
    \coordinate (T) at (60:\L);
    \coordinate (I) at (60:0.5*\L);

    % MEDIUM
    \draw[line width=0.8,blue]  (I)++(150+\angi:0.3*\L) -- (I) --++ (\angi-30:0.01*\L);
    \draw[line width=0.6,white] (I)++(150+\angi:0.3*\L) -- (I) --++ (\angi-30:0.01*\L);
    \fill[glass] (O) -- (T) -- (R) -- cycle;
    \path[name path=side] (T) -- (R);

    % LIGHT BEAMS
    % https://tex.stackexchange.com/questions/230227/creating-a-rainbow-color-macro
    \begin{scope}
        \coordinate (IT) at ($(I)+(60:0.006*\L)$);
        \clip (O) -- (T) -- (1.08*\L,0.45*\L) to[out=-35,in=20,looseness=1]
        (1.04*\L,0.2*\L) -- (R) -- cycle;
        \foreach \i [evaluate={
            \f=\i/\N;
            \ng=1.6-\f*0.2;
            \lamb=410+\f*320;
            \angr=asin(\na/\ng*sin(\angi));
            \angR=asin(\ng/\na*sin(60-\angr));
            \dl=0.0108*(1-\f)*\L;}] in {0,...,\N}{
            \definecolor{tmpcol}{wave}{\lamb}
            \colorlet{mycol}[rgb]{tmpcol}
            \path[name path=beam1] (I) --++ (-30.1+\angr:0.8*\L);
            \path[name path=beam2] (I) --++ (-30.0+\angr:0.8*\L);
            \coordinate (IT2) at ($(I)+(57:0.005*\L)+(-120:\dl)$);
            \fill[mycol,name intersections={of=side and beam1,name=t},
            name intersections={of=side and beam2,name=b}] %wave={\len}
            (IT2) -- (t-1) -- ($(t-1)+(30.0-\angR:0.4*\L)$)
            -- ($(b-1)+(30.1-\angR:0.4*\L)$) -- (b-1) -- ($(IT2)+(-120:-0.001*\L)$); %node[scale=0.3] {\angR};
        }
    \end{scope}
    \fill[white,path fading=east]
    (IT) --++ (-120:0.012*\L) --++ (-1.3:0.06*\L) --++ (80:0.015*\L) -- cycle;
    \fill[white,path fading=east]
    (IT) --++ (-120:0.012*\L) --++ (-1.3:0.10*\L) --++ (80:0.018*\L) -- cycle;

\end{tikzpicture}
		\caption{Дисперсія в призмі}
		\label{pic:Prism_dispersion}
	\end{subfigure}
	\begin{subfigure}{0.45\linewidth}
		% DROPLET dispersion & rainbow
\colorlet{watercol}{blue!70!cyan!50}
\tikzstyle{water}=[ball color=watercol]
\tikzset{
    beam/.style={very thick,line cap=round,line join=round},
}
\begin{tikzpicture}
    \def\L{1.8}            % length of ray outside droplet
    \def\R{1.6}            % droplet radius
    \def\na{1.0}           % air
    \def\N{6}              % number of rays
    \def\alphI{100}        % A: incident (180-90)
    \pgfmathsetmacro\thetI{180-\alphI} % theta_1: incident

    % WATER DROPLET
    \fill[water] (O) circle (\R);
    \fill[watercol!20,opacity=0.8] (O) circle (\R);
    \draw[blue] (O) circle (\R);

    % LIGHT RAYS
    \coordinate (O) at (0,0);
    \coordinate (A) at (-\L-\R,{\R*sin(\alphI)}); % incident ray
    \foreach \i [evaluate={
        \lamb=410+\i*320/\N;              % wavelength (for color)
        \nw=1.36-0.08*\i/\N;              % refractive index of water
        \thetII=asin(\na/\nw*sin(\thetI); % theta_2: air -> water & reflection
        \alphII=\alphI+2*\thetII-180;     % C: reflected
        \alphIII=\alphI+4*\thetII-360;    % D: exiting
        \s=0.8+0.45*\i/\N;                % scale
    }] in {0,...,\N}{
        \definecolor{tmpcol}{wave}{\lamb}
        \colorlet{mycol}[rgb]{tmpcol}
        \coordinate (B) at (\alphI:\R);                     % entry of incident
        \coordinate (C) at (\alphII:\R);                    % internal reflection
        \coordinate (D) at (\alphIII:\R);                   % exit of ray
        \coordinate (E) at ($(D)+(\alphIII-\thetI:\s*\L)$); % final ray to observer
        \draw[beam,thick,mycol]
        (B) -- (C) -- (D) -- (E);
    }

    % WHITE FADE
    \def\nw{1.30}
    \pgfmathsetmacro\thetII{asin(\na/\nw*sin(\thetI)}
    \draw[myblue,line width=1.3] (A) -- (B);
    \draw[beam,white,line width=1.0] (A) -- (B);
    %\draw[beam,white,line width=1.0,path fading=east] (B) --++ (-180+\alphI+\thetII:0.01);
    \draw[beam,white,line width=1.0,path fading=east] (B) --++ (-180+\alphI+\thetII:0.6*\R);
    %\draw[beam,white,line width=1.2] (B)++(-0.005,0) -- (B) --++ (-180+\alphI+\thetII:0.005);

\end{tikzpicture}
		\caption{Дисперсія в дощовій краплі}
		\label{pic:Droplet_dispersion}
	\end{subfigure}
	\caption{Прояви дисперсії}
	\label{pic:dispersion}
\end{figure}
%---------------------------------------------------------

Дисперсійна формула Коші, що застосовується для апроксимації
експериментальних залежностей $n(\lambda)$, має вигляд
\begin{equation}\label{eq:Cauchi_dispersion}
	n = A + \frac{B}{\lambda^2} + \frac{C}{\lambda^4} + \ldots
\end{equation}
де $A$, $B$, $C$ --- сталі величини, що визначаються для кожної речовини
дослідним шляхом; $\lambda$ --- довжина хвилі у вакуумі.




%%---------------------------------------------------------
%\subsection{Класична теорія дисперсії}
%%---------------------------------------------------------


%Згідно класичної теорії дисперсія світла виникає внаслідок вимушених коливань електронів під впливом змінного поля електромагнітної хвилі, яка взаємодіє з речовиною.
%Тому для пояснення дисперсії необхідно в модель ввести уявлення про структурі атомів чи молекул.

%Розглянемо розріджені речовини, у яких взаємодією між атомами або молекулами можна знехтувати
%(наприклад, гази). В цьому випадку, взаємодію поля з речовинною можна моделювати як вплив поля хвилі на ізольований атом, який спрощено розглядається як гармонічний осцилятор (рис.~\ref{pic:opt_electron}). Розрахунок такого впливу можна аналізувати за допомогою класичної механіки.

%---------------------------------------------------------
%\begin{wrapfigure}{O}{0.55\linewidth}\centering
%    \colorlet{myred}{red!65!black}
\tikzstyle{spring}=[line width=0.8,blue!7!black!80,snake=coil,segment amplitude=5,segment length=5,line cap=round]

\tikzstyle{force}=[->,myred,very thick,line cap=round]

\begin{tikzpicture}
    \node[circle, ball color=blue, minimum size=1.5cm] (O) at (0,0) {$+$};
    \node[circle, ball color=red, minimum size=0.2cm] (e) at (4,0) {$e$};
    \draw[spring,segment length=5] (O) -- (e);
    \draw[force] (e) --++ (2,0) node[pos=0.9,right=0] {$e\vb{E}$};
    \draw[force] ([yshift=-0.1cm]e.north) --++ (-2,0) node[pos=0.9,above=0] {$-k\vb{r}$};
    \draw[force] ([yshift=+0.1cm]e.south) --++ (-1,0) node[pos=0.9,below=0] {$-g\dot{\vb{r}}$};
    \draw[black,->] ([yshift=+0.1cm]e.south) --++ (+1,0) node[pos=0.9,below=0] {$\dot{\vb{r}}$};
\end{tikzpicture}

%    \caption{Модель оптичного електрона}
%    \label{pic:opt_electron}
%\end{wrapfigure}
%---------------------------------------------------------


%Зрозуміло, що застосування до опису руху електрона законів класичної механіки з точки
%зору сучасної фізики не виправдано. Правильну теорію атома дає квантова механіка. Однак, використання спрощеної моделі грунтується лише на тому факті, що квантова теорія дисперсії призводить практично до  тих же результатів, що й класична, хоч і з деякими особливостями.
%
%Усі електрони в атомі можна розділити на зовнішні (оптичні) та внутрішні електрони,
%тобто електрони внутрішніх оболонок. Для різних довжин хвиль різні електрони дають внесок у дисперсію. У
%оптичному діапазоні внесок дають лише зовнішні електрони. У класичній теорії дисперсії оптичний
%електрон (електрон зовнішньої оболонки атома) в атомі розглядається як згасаючий гармонічний
%осцилятор, що характеризується певною власною частотою $\omega_0$ і постійної згасання $\gamma$.
%
%Запишемо рівняння Ньютона, для руху електрона в полі електромагнітної хвилі, нехтуючи магнітною складовою сили Лоренца:
%\begin{equation}\label{eq:NewtEqn}
%    m\ddot{\vect r} = -k\vect{r} - g\dot{\vect{r}} - e\vect{E},
%\end{equation}
%де $\vect r$ --- зміщення електрона з положення рівноваги, $-k\vect{r}$ --- квазіпружна сила;  $-g\dot{\vect{r}}$ --- аналог сили тертя; $\vect{E}$ ---  напруженість електричного поля, що діє
%на електрон.
%
%Отже, рівняння \eqref{eq:NewtEqn} руху оптичного електрона в змінному полі $\vect{E}(t)$ електромагнітної хвилі приймає вигляд рівняння вимушених коливань гармонічного осцилятора:
%\begin{equation}\label{eq:Oscillations}
%    \ddot{\vect r} + 2\gamma\dot{\vect{r}} + \omega_0^2 \vect{r} = \frac{e}{m}\vect{E}(t),
%\end{equation}
%де параметри коливань: $\gamma =  \frac{g}{2m}$ --- коефіцієнт згасання та $\omega_0^2 = \frac{k}{m}$ --- власна частота коливань оптичного електрона. Величини $\gamma$ та $\omega_0$ не можуть бути отримані з класичної фізики, тут вони вводяться як параметри.
%
%Коефіцієнт згасання $\gamma$ містить внесок, зумовлений радіаційним тертям. Інші внески в цей коефіцієнт --- взаємодія з іншими атомами та непружні зіткнення пов'язані з дисипацією енергії електромагнітного поля. Відносна роль окремих членів рівняння залежить від області частот.
%Наприклад, при частотах $\omega$ падаючої електромагнітної хвилі, що далека від власної частоти $\omega_0$
%осцилятора, загасанням, як правило, можна знехтувати. Власну частоту $\omega_0$ атомного електрона
%будемо розглядати як формально введену постійну, яка визначає частоту лінії поглинання
%у спектрі досліджуваної речовини.
%
%
%Отже, нехай на осцилятор падає плоска монохроматична хвиля, напруженість якої зручно представити у комплексній формі:
%\begin{equation*}
%    \hat{\vect{E}}(\vect{r}, t) = \vect{E}_0 e^{i(\omega t - \vect{k}\cdot\vect{r})}.
%\end{equation*}
%
%
%$\vect{E}_0$ можна вважати постійною (не залежною від координат), якщо амплітуда коливань
%електрона мала порівняно з довжиною хвилі. Нас цікавить частковий розв'язок рівняння \eqref{eq:Oscillations}, що описує встановлені вимушені коливання осцилятора.
%Ці коливання, які здійснюються під дією гармонійної сили, також будуть
%гармонійними, частота яких збігатиметься з частотою вимушуючої сили. Тому розв'язок рівняння для
%$\vect{r}$ шукатимемо у вигляді
%\begin{equation*}
%    \hat{\vect{r}} = \hat{\vect{r}}_0e^{i(\omega t}.
%\end{equation*}
%Підставимо останнє рівняння в \eqref{eq:Oscillations} і отримаємо:
%\begin{equation}\label{eq:r_solution}
%    \hat{\vect{r}}(t) = \frac{1}{\omega_0^2 - \omega^2 - i\omega\gamma} \frac{e}{m}\hat{\vect{E}}(t).
%\end{equation}
%
%Оскільки електрон зміщується відносно ядра, то атом набуває дипольного моменту $\vect{p} = e\vect{r}$, а речовина в цілому поляризується $\vect{P} = N\vect{p}$, де $\vect{P}$ --- вектор поляризації, $N$ --- концентрація атомів. З іншого боку, індукована поляризація $\vect{P} = \chi \vect{E}$ (СГС), де $\chi$ ---  поляризовність речовини, яка пов'язана з діелектричною проникністю за формулою:
%\begin{equation*}
%    \hat{\epsilon}(\omega)  = 1 + 4\pi\hat{\chi}, \ \text{(СГС)}
%\end{equation*}
%Отже,  \eqref{eq:r_solution} можна записати з використанням вищезазначених величин і виразити~$\epsilon$:
%\begin{equation}\label{eq:epsilon}
%    \hat{\epsilon}(\omega)  = 1 + \frac{4\pi N e^2}{m}\frac{1}{\omega_0^2 - \omega^2 - i\omega\gamma}.\ \text{(СГС)}
%\end{equation}
%Для спрощення запису вводять константу
%\begin{equation}\label{eq:Lengmur}
%    \omega_p^2 = \frac{4\pi Ne^2}{m},\  \text{(СГС)}
%\end{equation}
%де $\omega_p$  має розмірність частоти і називається \emph{плазмовою} або \emph{ленгмюрівською частотою}.

Згідно класичної теорії, дисперсія світла виникає внаслідок вимушених коливань зовнішні (оптичних) електронів під впливом змінного поля електромагнітної хвилі, яка взаємодіє з речовиною. Теорія дисперсії розріджених середовищ у яких атоми мають лише один оптичний електрон дає вираз для комплексної діелектричної проникності:


\begin{equation}\label{eq:epsilon}
	\hat{\epsilon}(\omega)  = 1 + \frac{4\pi N_e e^2}{m_e}\frac{1}{\omega_0^2 - \omega^2 - 2i\omega\gamma} = 1 + \frac{\omega_p^2}{\omega_0^2 - \omega^2 - 2i\omega\gamma}, \ \text{(СГС)}
\end{equation}
де $N_e$ --- концентрація електронів, $e$ --- заряд електрона, $m_e$ --- маса електрона, $ \omega $~--- частота падаючої хвилі, $ \omega_0 $ --- власна частота коливань оптичного електрона, $ \gamma $ --- коефіцієнт згасання, $\omega_p$ --- \emph{плазмова} (\emph{ленгмюрівська}) частота, яка дається виразом:
\begin{equation}\label{eq:Lengmur}
	\omega_p^2 = \frac{4\pi N_e e^2}{m_e},\  \text{(СГС)}
\end{equation}


Згідно рівнянь Максвелла:
\begin{equation}\label{eq_dispersio:e=n^2}
	\hat{\epsilon} = \hat{n}^2
\end{equation}

Комплексний показник заломлення представляють у вигляді:
\begin{equation}\label{eq_dispersion: n}
	\hat{n} =  n  - i \chi
\end{equation}

Виділяючи дійсну та уявну частину з \eqref{eq:epsilon} можна отримати вирази:
\begin{equation}\label{eq:n-e_theory}
	n^2 - \chi^2 = 1 + \frac{\omega_p^2(\omega_0^2 - \omega^2)}{(\omega_0^2 - \omega^2)^2 + 4\omega^2\gamma^2}.
\end{equation}
та
\begin{equation}\label{eq:ne_theoty}
	n\chi = \frac{\omega_p^2 \omega\gamma}{(\omega_0^2 - \omega^2)^2 + 4\omega^2\gamma^2}
\end{equation}

З цих співвідношень можна отримати показники заломлення $n$ та коефіцієнт екстинкції $\chi$.

При частотах, далеких від власної частоти $\omega_0$ виконується умова
\begin{equation}\label{eq:prozorost}
	|\omega^2 - \omega_0^2| \gg 2\omega\gamma.
\end{equation}
%яка називається \emphz{умовою прозорості речовини}.
%, завдяки тому, що в цьому випадку рівняння  \eqref{eq:ne_theoty} дає $n\eta \approx 0$, тобто $\eta \approx 0$ (нема поглинання світла, отже --- речовина прозора).
Для розріджених прозорих речовин \eqref{eq:n-e_theory} з урахуванням \eqref{eq:prozorost} дає:
\begin{equation}\label{key}
	n \approx 1 + \frac12 \frac{\omega_p^2}{\omega_0^2 - \omega^2}.
\end{equation}

За умови $\omega_0 - \gamma \lesssim \omega_0 \lesssim \omega_0 + \gamma$ показник заломлення зростає із збільшенням частоти (рис.~\ref{plt:n,a(omega)}). Такий характер залежності $n(\omega)$, як було зазначено, називають \emphz{нормальною дисперсією}.




%Досліджуючи дисперсію рентгенівських променів у широкому діапазоні частот, вдається спостерігати і аномальну дисперсію цих променів, що дозволяє експериментально знаходити власні частоти коливань електронів (в атомах речовини), які пов'язані з атомом більш жорсткіше, ніж звичайні оптичні електрони.


%Коли $\omega \approx \omega_0$ показник загасання $\eta$ як функція частоти
%досягає максимального значення (рис.~\ref{plt:n,a(omega)}), тобто при цій умові світло частоти
%$\omega$ інтенсивно поглинається, а отже, середовище стає непрозорим. В цій області показник заломлення зменшується зі зростанням частоти $\frac{dn}{d\omega} < 0$, тому ця область є областю \emphz{аномальної дисперсії}.

%При цьому, показник заломлення зменшується зі зростанням частоти $\frac{dn}{d\omega} < 0$ ($\frac{dn}{d\lambda} > 0$), що відбувається в межах частот ($\omega = \omega_0 - \gamma \div \omega + \gamma$). Тому, ця область є областю аномальної дисперсії. Експериментально ця область була
%відкрита Ф.~Леру в 1860 р. у дослідах із заломленням білого світла призмою, наповненою парами йоду.
%Виявилося, що сині промені заломлюються менше ніж червоні, тобто, показник заломлення зменшується з частотою.

\begin{Attention}\small

	Для низьких частот ($\omega < \omega_0$) показник заломлення більше одиниці $n > 1$, тобто, фазова швидкість $v = \frac{c}{n} < c$ хвилі в середовищі менші за швидкість світла у вакуумі. Це означає, що хвиля в середовищі відстає від падаючої хвилі по фазі. Якщо ж частота світла більша за власну частоту осциляторів ($\omega > \omega_0$), то $n < 1$ і фазова швидкість хвилі в середовищі $v = \frac{c}{n} > c$ виявляється більше швидкості світла в вакуумі, тобто,  хвиля в середовищі випереджає по фазі падаючу.

	Цей теоретичний висновок добре узгоджується з дослідними результатами. Так, наприклад, для скла при довжині хвилі близько $0,1$~нм (рентгенівські хвилі) одержано $n = 0,999999$, тому для них можна спостерігати явище повного внутрішнього відбивання на межі повітря-скло.
\end{Attention}

%---------------------------------------------------------
\begin{figure}[h!]\centering
	\begin{tikzpicture}
    \def\ymax{2.1}
    \def\xmax{2.5}
    \def\ymin{0}
    \def\xextremaF{0.8269}
    \def\xextremaS{1.18}
    \begin{axis}[trim axis left, %trim axis right,
        height=0.5\linewidth,
        width=0.9\linewidth,
        /pgf/number format/.cd,
        use comma,
        1000 sep={\,},
        xlabel = {$\omega/\omega_0$},
%        ylabel style={text width=4.5cm},
        ylabel = \empty,
        %        grid=both,
        %        grid style={line width=.1pt, draw=blue!10},
        %        major grid style={line width=.2pt,draw=blue!50},
        minor tick num = 4,
        xtick distance=0.5,
        %        ytick distance=0.2,
        xmin=0,
        xmax=\xmax,
        ymax=\ymax,
        ymin=\ymin,
        %        ytick={1},
        extra x ticks={\xextremaF, 1, \xextremaS},
        extra x tick labels={$1 - \frac{\gamma}{\omega_0}$, 1, $1 +\frac{\gamma}{\omega_0}$},
        extra tick style={grid=major, grid style={dashed, black}},
        ]
        \addplot[mark=none, samples=500, domain=0:\xmax, red, thick] {1 + 0.5*(1-x^2)/((1-x^2)^2 + 0.4^2)};
        \addlegendentry{$n$ --- показник заломлення}
        \addplot[mark=none, samples=500, domain=0:\xmax, blue, thick] {0.3/((1-x^2)^2 + 0.4^2)};
         \addlegendentry{$\chi$ ---  коефіцієнт екстинкції}
        \addplot[mark=none, black, samples=2, dashed] {1};
        \path[name path=A] (\xextremaF, \ymin) -- (\xextremaF,\ymax);
        \path[name path=B] (\xextremaS, \ymin) -- (\xextremaS,\ymax);
%
        \addplot [red!30, opacity=0.5] fill between [of = A and B];


        \node[text width=4cm, align=center, font=\footnotesize, text=black] at (0.4, 1.8) {Область \\ нормальної дисперсії \\ $\frac{dn}{d\omega}  > 1$, $n > 1$};

        \node[text width=4cm, align=center, font=\footnotesize\itshape, text=red!90!black] (ADsignature) at (1.8, 1.4)  {Область \\ аномальної дисперсії\\ $\frac{dn}{d\omega}  < 1$};

        \node[text width=4cm, align=center, font=\footnotesize, text=black]  at (2, 0.5) {Область \\ нормальної дисперсії \\ $\frac{dn}{d\omega}  > 1$, $n < 1$};
        \coordinate (AD) at (\xextremaS,1.9);
        \draw[->] (ADsignature.west) to[in=0, out=180] (AD);
    \end{axis}
\end{tikzpicture}
	\caption{Криві функцій для показника заломлення $n$ та коефіцієнта екстинкції $\chi$ поблизу резонансної частоти $\omega_0$}
	\label{plt:n,a(omega)}
\end{figure}
%---------------------------------------------------------

Для плазми, яка є іонізованим газом, а також для електронного газу в металах, в якому власні частоти вільних електронів дорівнюють нулю, а коливанням
важких іонів можна знехтувати ($\frac{m_e}{m_n} \ll 1$, $m_n$ --- маса додатного іону) ---
діелектрична проникність визначається головним чином
вільними електронами. Тому, поклавши в \eqref{eq:epsilon} $\omega_0 = 0$, для плазми отримаємо:
\begin{equation}\label{eq:plasm}
	\epsilon = 1 - \left( \frac{\omega_p}{\omega}\right)^2.
\end{equation}

Зазвичай, $\omega < \omega_p$ (приклад радіохвилі ультракороткого діапазону, що падають на іоносферу Землі). В цьому випадку $n$ є суто уявною
величиною (не має дійсної частини), тому з рівняння \eqref{eq_dispersion: n} випливає, шо для плазми дійсна $\Re{(\hat{n})} = 0$. В цьому випадку
плазма (або метал) є непрозорою для електромагнітних хвиль. Хвилі при падінні на її поверхню зазнають повного внутрішнього відбивання. При $\omega >
\omega_p$ уявна частина $\Im{(\hat{n})} = 0$ і хвилі в розрідженій плазмі, або металі поширюються майже без поглинання ($\omega \gg \gamma$). Плазма
(або метал є повністю прозорими для електромагнітної хвилі. Границею прозорості є умова $\epsilon = 0$, яка досягається за випадку $\omega =
\omega_p$.


%Зрозуміло, наведена вище теорія враховувала лише електрони з однією власною частотою $\omega_0$. Однак в атомах є й інші електрони з іншими власними частотами, тому діелектрична проникність \eqref{eq:epsilon} в цьому випадку повинна мати вигляд:
%\begin{equation*}
%    \hat{\epsilon}(\omega)  = 1 +\sum\limits_i \frac{4\pi N_i e^2}{m}\frac{1}{\omega_{0_i}^2 - \omega^2 - i\omega\gamma_i}.
%\end{equation*}



%%% --------------------------------------------------------
\subsection{Фазова та групова швидкості}
%%% --------------------------------------------------------



Монохроматична хвиля, за визначенням, є нескінченна у просторі та у часі. Реальна хвиля завжди просторово обмежена і випромінюється протягом обмеженого інтервалу часу, а тому не є строго монохроматичною. Проте будь-яку реальну хвилю можна як результат суперпозиції великої кількості строго монохроматичних плоских хвиль. Тобто реальна хвиля є групою монохроматичних складових.

\begin{figure}[h!]\centering
	\begin{tikzpicture}
		\draw[red, samples=1000] plot (\x, {2*exp(-0.25*\x*\x)*cos(deg(10*\x))});
		\draw[blue, samples=1000] plot (\x, {2*exp(-0.25*\x*\x)});
		\draw[blue, samples=1000] plot (\x, {-2*exp(-0.25*\x*\x)});
		\draw[->] (0, {2*exp(-0.25*0)}) -- ++(1.5,0) node[above] {$\vect{u}$};
	\end{tikzpicture}
	\caption{Приклад пакету хвиль}
\end{figure}

Під фазовою швидкістю розповсюдження хвиль розуміють швидкість, з якою поширюється
поверхня однакових фаз:
\begin{equation}\label{eq:pahse_velocity}
	v = \frac{\omega}{k}
\end{equation}


За відсутності дисперсії фазова швидкість хвиль не залежить від частоти. Тому, якщо є група хвиль різних частот, то всі вони будуть рухатися з однією і тією ж швидкістю і пакет, який вони утворюють у
в результаті накладання, при русі не змінює своєї форми, його огинаюча рухатиметься з тією ж швидкістю, що і хвилі, з яких складається.

Якщо дисперсія є, то огинаюча рухається з іншою швидкістю, ніж швидкості монохроматичних компонент.

Швидкість руху максимуму амплітуди групи хвиль або хвильового пакету ---  називається груповою швидкістю.

%\begin{Attention}
%    Фазова швидкість може бути більше швидкості світла, і це не суперечить релятивістській теорії, яка стверджує, що швидкість матеріальних тіл та швидкість сигналу не можуть перевищувати $c$.
%
%    Однак, максимум інтенсивності (енергії) припадає на максимум огинаючої у хвильовому пакеті
%    Тому в тих випадках, швидкість перенесення енергії (і інформації) хвилею дорівнює груповий швидкості і не перевищує $c$ у вакуумі.
%\end{Attention}

Групова швидкість визначається за формулою
\begin{equation}\label{eq: group_velocity}
	u = \frac{d\omega}{dk}.
\end{equation}

Як легко бачити, зв'язок фазової і групової швидкості маж вигляд:
\begin{equation}\label{eq:Reley}
	u = v- \lambda \frac{dv}{d\lambda},
\end{equation}
яке носить назву формули Релея.

\begin{Attention}
	Фазова швидкість може бути більше швидкості світла, і це не суперечить релятивістській теорії, яка стверджує, що швидкість матеріальних тіл та швидкість сигналу не можуть перевищувати $c$.

	Групова швидкість в залежності від знаку доданку $\frac{dv}{d\lambda}$  може бути меншою, так і більше
	фазової, але завжди менше швидкості світла у вакуумі. Коли дисперсії немає (у вакуумі) $u = v$.

	Максимум інтенсивності (енергії) припадає на максимум огинаючої у хвильовому пакеті
	Тому в тих випадках, швидкість перенесення енергії (а отже, і інформації) хвилею дорівнює груповий швидкості і не перевищує $c$ у вакуумі.

	Поняття групової швидкості не застосовується, коли поглинання середовища дуже велике (область аномальної
	дисперсії).
\end{Attention}




%%% --------------------------------------------------------
\section{Приклади розв’язування задач}
%%% --------------------------------------------------------


\Example{Показник заломлення прозорої речовини для деякого
	спектрального діапазону описується двочленною дисперсійною формулою
	Коші:
	\begin{equation*}
		n = A + \frac{B}{\lambda^2},
	\end{equation*}
	де $A > 0$, $B > 0$.
	Визначити: а) дисперсію показника заломлення речовини; б) фазову
	швидкість світла; в) групову швидкість світла.
}

\begin{solutionexample}[height fill=false]
	а) Дисперсія показника заломлення речовини
	\begin{equation*}
		\frac{dn}{d\lambda} = - \frac{2B}{\lambda^3}.
	\end{equation*}

	б) Фазова швидкість світла у речовині

	\begin{equation*}
		v = \frac{c}{n} = \frac{c\lambda^2}{A\lambda^2 + B}.
	\end{equation*}

	в) Групова швидкість світла у речовині визначається через фазову за формулою Релея:
	\begin{equation*}
		u = v - \lambda \frac{dv}{d\lambda} = \frac{C\lambda^2(A\lambda^2 - B)}{A\lambda^2 + B}.
	\end{equation*}
\end{solutionexample}


%\Example{Показник заломлення повітря за нормальних умов ($t = 0^\circ$C, $p = 0,1013$~МПа) для жовтої лінії натрія дорівнює $1,0002918$. Визначити показник заломлення повітря при температурі $30^\circ$C та тиску $3\cdot10^6$~Па}

%\begin{solutionexample}
%Оскільки лінії поглинання у спектрах основних газів повітря --- азоту
%та кисню --- не співпадають з лінією випромінювання натрію,
%використаємо для оцінки $n$ формулу Лоренца:
%\begin{equation*}
%    \frac{n^2 - 1}{n^2 + 2} = \frac{4 \pi}{3} N \alpha,
%\end{equation*}
%де $n$ --- показник заломлення,  $N$ ---  концентрація молекул, $\alpha$ --- їх поляризовність.

%Згідно формули \href{https://en.wikipedia.org/wiki/Clausius%E2%80%93Mossotti_relation#Lorentz%E2%80%93Lorenz_equation}{Лоренца-Лоренца}, показник заломлення газуів пропорційний
%Вважаючи, що зі зміною температури та тиску змінюється тільки
%концентрація електронів в одиниці об’єму повітря $N$, отримуємо
%співвідношення
%\begin{equation*}
%
%\end{equation*}
%\end{solutionexample}

\Example{Напишіть рівняння плоскої хвилі для випадків  коли комплексний показник заломлення дорівнює $\hat{n} = n - i\chi$ та $\hat{n} = - i\chi$, відповідно. З'ясуйте фізичний зміст таких показників заломлення.}

\begin{solutionexample}
	Рівняння плоскої хвилі, що поширюється вздовж додатного напрямку осі $x$ в середовищі запишемо у вигляді:
	\begin{equation}\label{eq_dispersion:E_Plane_wave_complex}
		\vect{E} = \vect{E}_0 e^{i(\omega t - \hat{k} x)},
	\end{equation}
	де $\hat{k} = \frac{2\pi\hat{n}}{\lambda_0}$ --- комплексне хвильове число, $\lambda_0$ --- довжина хвилі у вакуумі, $\hat{n}$ --- комплексний показник заломлення середовища.

	Якщо $\hat{n} = n - i\chi$, то \eqref{eq_dispersion:E_Plane_wave_complex} приймає вигляд:
	\begin{equation*}
		\vect{E} = \vect{E}_0e^{-\frac{2\pi \chi}{\lambda_0} x} e^{i(\omega t - \frac{2\pi n}{\lambda_0} x)} = \vect{A} e^{i(\omega t - \frac{2\pi n}{\lambda_0} x )},
	\end{equation*}
	що являє собою рівняння плоскої хвилі з амплітудою, яка зменшується за законом:
	\begin{equation*}
		\vect{A} = \vect{E}_0e^{-\frac{2\pi \chi}{\lambda_0} x}  = \vect{E}_0e^{- \frac{k_{\omega}}{2} x},
	\end{equation*}
	де $k_{\omega} = \frac{4\pi\chi}{\lambda_0}$ --- показник поглинання хвилі середовищем, який має розмірність см$^{-1}$.

	Оскільки інтенсивність хвилі $I \sim \vect{A}^2$, то з останнього рівняння отримуємо вираз:
	\begin{equation*}
		I = I_0e^{- k_{\omega} x},
	\end{equation*}
	який називається \emphz{законом Бугера-Ламберта-Бера}.

	При $\hat{n} = - i\chi$, \eqref{eq_dispersion:E_Plane_wave_complex} приймає вигляд:
	\begin{equation}\label{eq_dispersin:E_stand_wave}
		\vect{E} = \vect{E}_0e^{-\frac{2\pi \chi}{\lambda_0} x} e^{i\omega t} = \vect{E}_0e^{-\frac{x}{\delta}} e^{i\omega t},
	\end{equation}
	що уже являє собою рівняння плоскої \emph{стоячої хвилі}, з амплітудою, що також як і в попередньому випадку, експоненційно спадає.

	\emph{Примітки}: З фізичної точки зору, рівняння вигляду \eqref{eq_dispersin:E_stand_wave} описує електричне поле в плазмі за умови $\omega < \omega_p$. В цьому випадку вводять поняття \emph{глибини проникнення} електричного поля хвилі в середовище, яка визначається як відстань, на якій амплітуда електричного поля спадає в $e$ разів, тобто:
	\begin{equation*}
		\delta = \frac{\lambda_0}{2\pi\chi}.
	\end{equation*}

	%\emph{Примітка}: При падінні хвилі на поверхню розділу двох середовищ
	%Проте, у цьому випадку світло зазнає повного внутрішнього відбивання у середовищі (без поглинання!).

	%Цікаво відмітити, що за умови $\hat{n} = - i\eta$, комплексний показник заломлення $\hat{\epsilon} = 0$ (див. \eqref{eq_dispersio:e=n^2} та \eqref{eq_dispersion: n}).
\end{solutionexample}

\Example{Знайти число вільних електронів на атом срібла, якщо плівка
	срібла прозора для ультрафіолетового випромінювання, починаючи з
	енергії $8,9$~еВ. Для срібла відносна атомна маса $A=108$, густина $\rho=10,5$~г/см$^3$.}

\begin{solutionexample}

	Фотону з енергією $8.9$~еВ відповідає довжина хвилі $\lambda = 1,393 \cdot 10^{-5}$~см.

	Умовою прозорості плазми у речовини на даних частотах є $\omega = \omega_p$ (див. рівн.\eqref{eq:plasm}).

	Оскільки $\omega_p^2 = \frac{4\pi N_e e^2}{m_e}$ (див. рівн. \eqref{eq:Lengmur}),
	\begin{equation*}
		\frac{4\pi N_e e^2}{m_e} = \omega^2 = \left( \frac{2\pi c}{\lambda} \right)^2.
	\end{equation*}

	Звідки виразимо концентрацію електронів вільних електронів срібла
	\begin{equation*}
		N_e = \frac{\pi c^2  m_e}{e^2 \lambda^2}.\  \text{(СГС)}
	\end{equation*}

	Концентрація атомів срібла:
	\begin{equation*}
		N = \frac{N_A \rho}{A}.
	\end{equation*}

	\pgfkeys{/pgf/fpu=true}
	\pgfmathsetmacro\wl{1.393e-5}
	\pgfmathsetmacro\c{3e10}
	\pgfmathsetmacro\e{4.8e-10}
	\pgfmathsetmacro\A{108}
	\pgfmathsetmacro\me{9.10958215e-28}
	\pgfmathsetmacro\density{10.5}
	\pgfmathsetmacro\NA{6.022e23}
	\pgfmathparse{(pi*\c^2*\me*\A)/(\e^2*\wl^2*\NA*\density)}
	\pgfmathsetmacro\renNeN{\pgfmathresult}
	\pgfkeys{/pgf/fpu=false}

	Число вільних електронів на один атом (значення підставимо в системі СГС):
	\begin{equation*}
		\frac{N_e}{N} = \frac{\pi c^2  m_e A }{N_A \rho e^2  \lambda^2} \approx \pgfmathprintnumber[fixed, precision=1]{\renNeN}.
	\end{equation*}

\end{solutionexample}


\Example{Знайти зв’язок між фазовою і груповою швидкостями електромагнітних хвиль у плазмі при $\omega > \omega_p$.}


\begin{solutionexample}[height fill=false]

	З формули \eqref{eq:plasm} для отримуємо вираз показника заломлення:
	\begin{equation*}
		n^2 = 1 - \left( \frac{\omega_p}{\omega}\right)^2.
	\end{equation*}

	З означення показника заломлення $v = \frac{c}{n}$, з іншого боку $v = \frac{\omega}{k}$, прирівняємо ці вирази $\frac{c^2}{n^2} = \frac{\omega^2}{k^2}$ і отримаємо функцію $\omega(k)$ (дисперсійне співвідношення):
	\begin{equation*}
		\omega^2 = \omega_p^2  - c^2k^2.
	\end{equation*}
	Продиференціюємо останній вираз:
	\begin{equation*}
		2\omega d\omega = 2c^2k dk
	\end{equation*}
	і поділимо праву і ліву частину на $2kdk$:
	\begin{equation*}
		\frac{\omega d\omega}{k dk} = c^2.
	\end{equation*}
	Враховуючи, що фазова швидкість $v = \frac{\omega}{k}$, а групова швидкість $u = \frac{d\omega}{dk}$, остаточно маємо:
	\begin{equation*}
		vu = c^2.
	\end{equation*}
\end{solutionexample}

%%% --------------------------------------------------------
\section{Задачі для самостійного розв’язку }
%%% --------------------------------------------------------

%=========================================================
\begin{problem}%
Що характеризує дисперсія показника заломлення речовини?
\end{problem}

%=========================================================
\begin{problem}%
Що характеризує дисперсія призми? Які значення має $\frac{dn}{d\lambda}$ нормальній та аномальній дисперсіях?
\end{problem}

%=========================================================
\begin{problem}%
Що характеризує групова швидкість? Як вона пов’язана з фазовою
швидкістю?
\end{problem}

%=========================================================
\begin{problem}%
Для яких речовин групова швидкість рівна або менше, або більше
фазової?
\end{problem}


%=========================================================
\begin{problem}%
Вивести залежність групової швидкості від дисперсії показника
заломлення речовини $\frac{dn}{d\lambda}$.
\end{problem}


%=========================================================
\begin{problem}%
Як виражається показник заломлення розрідженого газу через частоту
світла поблизу лінії поглинання? Накресліть приблизний графік цієї
залежності.
\end{problem}


%=========================================================
\begin{problem}%
Як виражається показник заломлення розрідженого газу через частоту
світла поблизу лінії поглинання? Накресліть приблизний графік цієї
залежності.
\end{problem}


%=========================================================
\begin{problem}%1.70
В дисперсійній формулі Коші для скла Ф1 в видимій області сталі
$A=1,5878$ і $B=0,0087$ (для довжини хвилі в мкм). Визначити для зеленої
лінії ртуті ($0,546$~мкм): а) показник заломлення скла Ф1; б) дисперсію
показника заломлення; в) фазову швидкість монохроматичного світла;
г) групову швидкість для групи хвиль з центром в $0,546$~мкм.
\begin{solution}
	а) $n_e=1.6169$; б) $\frac{d n}{d\lambda} =-0,1069$~м$^{-1}$ в) $u = 1,85\cdot10^{10}$~см/с, $v = 1,79\cdot10^{10}$~см/с.
\end{solution}
\end{problem}


%=========================================================
\begin{problem}%1.71
Для рівносторонньої призми з скла ТК12 експериментально визначені
мінімальні кути відхилення $43^\circ57'39''$, $43^\circ32'2''$ та $43^\circ7'5''$ для променів
трьох довжин хвиль відповідно $\lambda_{F'}= 0,480$~ мкм, $\lambda_e = 0,546$~мкм і
$\lambda_{C'}=0,6438$~мкм. Обчислити: а) показники заломлення $n_{\lambda}$ ; б) середню
дисперсію показника заломлення скла; в) середню кутову дисперсію
призми; г) коефіцієнти двочленної формули Коші; д) фазову та групову
швидкості світла у склі ТК12 (для$\lambda_e$).
\begin{solution}
	а) $n_{f'} = 1,5756$, $n_e = 1,5710$ , $n_{C'} =  1,5665$; б) $\frac{\Delta n}{\Delta\lambda} = 0,055$~м$^{-1}$ в) $\frac{\Delta \sigma_0}{\Delta\lambda} = 0,0898$~рад$\cdot$м$^{-1}$; г) $A = 1,5551$, $B = 0,0047$~мкм$^2$; д) $v_e = 1,91\cot10^{10}$~см/с, $u = 1,87\cdot10^{10}$~ см/с.
\end{solution}
\end{problem}


%=========================================================
\begin{problem}% 1.72
Знайти залежність між груповою $u$ і фазовою $v$ швидкостями для
наступних законів дисперсії: а) $v \sim \frac{1}{\sqrt{\lambda}}$, б) $v \sim k$, в) $v \sim \frac1{\nu^2}$,
де $\lambda$, $k$, $\nu$ --- довжина хвилі, хвильове число та часто, відповідно.
\end{problem}


%=========================================================
\begin{problem}%
Виведіть співвідношення \eqref{eq:n-e_theory} та \eqref{eq:ne_theoty}.
\begin{solution}
	\emph{Підказка}: Використайте формули \eqref{eq:epsilon}, \eqref{eq_dispersio:e=n^2} та \eqref{eq_dispersion: n}.
\end{solution}
\end{problem}

%=========================================================
\begin{problem}%
При зондуванні розрідженої плазми радіохвилями різних частот
виявили, що радіохвилі з $\lambda > 0,75$~м зазнають повного внутрішнього
відображення. Знайти концентрацію вільних електронів у цій плазмі.
\begin{solution}
	$N = 2\cdot10^9$ см$^{-3}$.
\end{solution}
\end{problem}


%=========================================================
\begin{problem}%1.73
Знайти концентрацію вільних електронів іоносфери, якщо для
радіохвиль з частотою $100$~МГц її показник заломлення $n=0,90$.
\begin{solution}
	$N = 2,4\cdot10^7$~см$^{-1}$.
\end{solution}
\end{problem}



%=========================================================
\begin{problem}%1.74
Маючи на увазі, що для жорсткого рентгенівського випромінювання
електрони речовини можна вважати вільними, визначити, на скільки
різниться з одиницею показник заломлення графіта для променів з
довжиною хвилі у вакуумі $\lambda=50$ нм.
\begin{solution}
	$n - 1 = - 5,5\cdot10^{-7}$
\end{solution}
\end{problem}



%=========================================================
\begin{problem}%1.75
Показник заломлення сірководню для світла з довжинами хвиль $509$,
$534$ і $589$~нм дорівнюють відповідно $1,647$, $1,640$ і $1,630$. Обчислити
фазову і групову швидкості світла поблизу $\lambda=534$~нм.
\begin{solution}
	$v = 0,61c$, $u = 0,65c$.
\end{solution}
\end{problem}















\Closesolutionfile{answer}


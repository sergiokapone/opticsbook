% !TeX program = lualatex
% !TeX encoding = utf8
% !TeX spellcheck = uk_UA

\clearpage
\section{Фізичні константи}

\begin{table}[h!]\centering
	\tabcaption{Деякі фізичні константи в системі СГС}
	\label{tab:SItoGauss}
	\small
	\begin{tblr}{
            colspec={lcl},
            cell{1}{3}= {c,m},
        }
		\toprule
		Константа                  & Символ   & {Значення}                           \\ \midrule
		Швидкість світла у вакуумі & $c$      & $2.997 924 58 \cdot 10^{10}$ см/с               \\
		Гравітаційна стала         & $G$      & $6.674 28 \cdot 10^{-8}$ см$^3$/(г$\cdot$c$^2$) \\
		Стала Планка               & $\hbar$  & $1.054 5716 \cdot 10^{-27}$ ерг$\cdot$с         \\
		Елементарний заряд         & $e$      & $4.803 204 27  \cdot 10^{-10}$ СГСЕ$_q $\footnotemark        \\
                                   &          & $1.60217662 \cdot 10^{-19}$ Кл (в SI)           \\
		Маса електрона             & $m_e$    & $9.109 382 15 \cdot 10^{-20}$ г                 \\
		Енергія спокою електрона   & $m_ec^2$ & $0.511 \cdot 10^{6}$ еВ                         \\
		Маса протона               & $m_p$    & $1.672 621 9 \cdot 10^{-30}$ г                  \\
		Енергія спокою протона     & $m_pc^2$ & $938.26 \cdot 10^{6}$ еВ                        \\
		\hskip4ex Електрон-Вольт   & еВ       & $1.602\cdot 10^{-12}$ ерг                       \\
		Борівський радіус          & $a_0$    & $5.291 772 0859 \cdot 10^{-9}$ см               \\
		Магнетон Бора              & $\mu_B$  & $9.274 009 15 \cdot 10^{-21}$ ерг/Гс            \\
		Стала Больцмана            & $k$      & $1.380 6504 \cdot 10^{-16}$ ерг/К               \\
		Універсальна газова стала  & $R$      & $8.314\cdot 10^{7}$ ерг/(К$\cdot$моль)          \\
		Число Авогадро             & $N_A$    & $6.022\cdot 10^{23}$  моль$^{-1}$               \\
		Атомна одиниця маси        & $u$      & $1.66042 10^{-24}$  г                           \\ \bottomrule
	\end{tblr}
\end{table}

\footnotetext{Міжнародна назва цієї одиниці~--- Франклін (позначення: Fr, Фр). У радянській і сучасній вітчизняній літературі зазвичай називається просто СГСЕ-од. заряду. Франклін є однією з чотирьох основних одиниць системи Гауса (поряд з сантиметром, грамом і секундою).}
% !TeX program = lualatex
% !TeX encoding = utf8
% !TeX spellcheck = uk_UA

\documentclass[
%border=1cm
]{ProblemBook}

\pagestyle{main}
%\usetikzlibrary{3d}


\usepackage{OpticsImages}




\begin{document}



%\input{interference/interference_lines.tikz}


%		\begin{tikzpicture}
%			\pgfmathsetmacro\N{6}
%			\pgfmathsetmacro\m{1}
%			\pgfmathsetmacro\n{\m*\N}
%			\pgfmathsetmacro\E{2}
%			\pgfmathsetmacro\dangle{pi*\m}
%			\draw[gray!50] (0,{\E+0.025}) circle ({\E+0.05});
%			\coordinate (@) at (0,0);
%
%			\foreach[count=\j] \i in {1,...,\n}
%			{
%
%			\draw[->] (@) -- coordinate[at end] (@) ++ ({(2*\i-1)*pi/2/\N r}:{(pi*\E/\N)});
%
%			\ifnum\j<2%
%				\draw[dashed] (@) -- ++({(2*\i-1)*pi/2/\N r}:0.7);
%				\draw (@) ++({(2*\i-1)*pi/2/\N r}:0.5)
%				arc[start angle={(2*\i-1)*pi/2/\N r}, delta angle={2*pi/2/\N r}, radius=0.5]
%				node[below right, font=\scriptsize] {$ \pi/\N $};
%                \node[below left, font=\scriptsize] at (@) { $ \Delta\vect{E} $};
%                \draw (0,\E) ++(0,-0.75) arc[start angle=-90, delta angle=30, radius=0.75] node[pos=0.7, font=\scriptsize, below] {$ \pi/6 $};
%                \draw[dashed] (0,\E) -- (@);
%			\fi
%			}
%			\draw[->, red] (0,0)  -- node[left] {$ \vect{E}_1 $} (@) ;
%		\end{tikzpicture}


\begin{figure}[h!]
    \centering
    \begin{tikzpicture}
        \node[circle, fill=red, inner sep=1pt] (S) at (0,0) {}; \node[left] at (S) {$S$};
        \node[circle, fill=red, inner sep=0.5pt] (P) at (6,0) {}; \node[right] at (P) {$P$};
        \begin{scope}
            \clip (S) circle (2);
            \foreach[count=\x] \i in {4,4.2,...,5.4}
            {
                \draw[gray!80, name path global=bigcrcle\x] (P) circle (\i);
            }
        \end{scope}
        \draw[ball color=yellow, opacity=0.3, name path global=smallcircle] (S) circle (2);
%        \draw (E) -- node[above] {$ r $} (P);
        \draw (S)  -- ++ (0:2) coordinate (P1)--  (P);
        \draw[decorate,decoration={brace,amplitude=5pt,mirror,raise=4pt},yshift=0pt] (P1) -- node[below=6pt] {$b$} (P);
        \draw[name intersections={of=smallcircle and bigcrcle5}] (intersection-1) -- node[above, sloped] {$ b + m\frac\lambda2 $} (P)
        (S) -- node[above] {$ R $} (intersection-1) coordinate(Top);
        \draw[decorate,decoration={brace,amplitude=5pt,mirror,raise=4pt},yshift=0pt] (S) -- node[below=6pt] {$a$}++(2,0);

        \draw[dashed, thick]
        %[decorate,decoration={brace,amplitude=5pt},yshift=0pt]
        (Top) -- node[pos=0.6, right] {$r_m$} ++(0,-1.42);

    \end{tikzpicture}
    \caption{Ілюстрація методу зон Френеля}
    \label{pic:Diffraction2}
\end{figure}


\end{document}



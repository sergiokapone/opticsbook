\protect \section *{\nameref *{Difraction}}
\begin{Solution}{5.{7}}
        а) $ 1 $~мм; б) $ 0,71 $~мм.
    
\end{Solution}
\begin{Solution}{5.{8}}
        \begin{tblr}%
            {
                colspec={lccccccc},
                hlines,
                vlines,
            }
            $m$       & 1     & 2   & 3     & 4  & 5     & 6  & 7     & 8 \\
            $b_m$, см & 200   & 100 & 66    & 50 & 40    & 33 & 28,6  & 25 \\
            $I_m$     & $I_0$ & 0   & $I_0$ & 0  & $I_0$ & 0  & $I_0$ & 0
        \end{tblr}
    
\end{Solution}
\begin{Solution}{5.{9}}
        $ 15 $ см.
    
\end{Solution}
\begin{Solution}{5.{10}}
        1) $ 9/16\cdot I_0 $; 2) $ 1/4\cdot I_0  $; 3) $ 1/16\cdot I_0 $; 4) $ 1/4\cdot I_0; $ 5) $ 25/16\cdot I_0 $; 6) $ 9/14\cdot I_0 $; 7) $ 49/16\cdot I_0 $; 8) $ 9/4\cdot I_0 $.
    
\end{Solution}
\begin{Solution}{5.{11}}
        1) $ I = 0 $ 2) $ I = 1/4\cdot I_0 $.
    
\end{Solution}
\begin{Solution}{5.{12}}
        $ I = 0 $.
    
\end{Solution}
\begin{Solution}{5.{13}}
        Інтенсивність збільшиться в $ 5 $ разів.
    
\end{Solution}
\begin{Solution}{5.{14}}
        Електричне поле в діелектрику максимальне на осі пучка на такій відстані від поверхні, з якої діаметр пучка сприймається як діаметр першої зони Френеля.
        \begin{equation*}
            b = \frac{D^2n}{4\lambda} = 100\ \text{см},
        \end{equation*}
        \begin{equation*}
            E = \frac43 \sqrt{\frac{8\pi S}{c}} = 1200\ \frac{\text{В}}{см}.
        \end{equation*}
    
\end{Solution}
\begin{Solution}{5.{15}}
        а) $ f' = 90 $~см; б) $ r_1 = 0,672 $~мм; в) $ f' =119,7 $~мм.
    
\end{Solution}
\begin{Solution}{5.{16}}
        \begin{equation*}
            f_{\max} = \frac{r_1^2}{\lambda} = 8\ \text{м}; \quad h = \frac{2m+1}{d2(n - 1)}\lambda, \quad m = 0, 1, 2, \ldots, \quad I_{\max} = 36 I_0.
        \end{equation*}
    
\end{Solution}
\begin{Solution}{5.{17}}
        \begin{equation*}
            r_m = \sqrt{\frac{m\lambda}{\left|\frac1a + \frac1b - \frac2R \right|}}.
        \end{equation*}
    
\end{Solution}
\begin{Solution}{5.{18}}
        а) $ m_1 = 40 $; б) $ \delta l_0 = 5 $ см.
    
\end{Solution}
\begin{Solution}{5.{19}}
        $d^{-1} = 500$ мм$^{-1}$; $ \lambda_2 = 0,4099 $~мкм.
    
\end{Solution}
\begin{Solution}{5.{20}}
         $ d = 5,013\cdot10^{-4} $~см; $ \lambda_2 = 0,706 $~мкм.
    
\end{Solution}
\begin{Solution}{5.{21}}
        а) Ні; б) Так
    
\end{Solution}
\begin{Solution}{5.{22}}
       $ 22,2 $~мкм.
    
\end{Solution}
\begin{Solution}{5.{23}}
        $ h = \frac{2m-1}{2(n-1)}\lambda$, $m = 1, 2, 3, \ldots$. Інтенсивність нульового головного максимуму
        дорівнює нулю.
    
\end{Solution}
\begin{Solution}{5.{24}}
        $I = I_0\left( \frac{\sin \left( kNa\sin\frac\theta2\right) }{\sin \left( ka\sin\frac\theta2\right) } \right)^2
        \left( \frac{\sin k\frac\Delta2}{k\frac\Delta2} \right)^2 $, де $k = \frac{2\pi}{\lambda}$, $\Delta = h (n-1) - a\sin\theta$. Напрямки на головні максимуми визначаються формулою: $\sin\theta_{\min} = \frac{m\lambda}{a}$.
    
\end{Solution}
\begin{Solution}{5.{25}}
        Роздільна здатність не зміниться, дисперсійна область зменшиться вдвічі.
    
\end{Solution}
\begin{Solution}{5.{26}}
        Для ока $4,5$~км; для труби --- $56,5$~км.
    
\end{Solution}
\begin{Solution}{5.{27}}
        $l = 52$~м; $\phi_d = 0,055''$.
    
\end{Solution}
\begin{Solution}{5.{28}}
        $\ell_{\min}\approx 1$~м.
    
\end{Solution}
\begin{Solution}{5.{29}}
        $\frac{D^2}{f^2} \ge z^2\lambda^2 \approx 0,25$.
    
\end{Solution}
\begin{Solution}{5.{30}}
        $\Gamma \ge \frac{D}{d}$.
    
\end{Solution}
\begin{Solution}{5.{31}}
        $\Gamma 2n  \frac{L}{d}\sin u$.
    
\end{Solution}
\begin{Solution}{5.{32}}
        $L \approx \frac{D^2}{2,44 \lambda} \approx 1000$~км; $ S = \left( 2,44\frac{\lambda}{D} \right)^2 = 1,5\cdot10^{-12}$.
    
\end{Solution}
\begin{Solution}{5.{33}}
        $L_{\max} \approx 70$~км.
    
\end{Solution}
\begin{Solution}{5.{34}}
        $\tau \approx 1,5$\%.
    
\end{Solution}
\begin{Solution}{5.{35}}
        $R_\text{ІЧ} = 2\cdot10^5$.
    
\end{Solution}
\begin{Solution}{5.{36}}
        $R_{\max} \approx \frac{fD}{\lambda d} = 5\cdot10^5$.
    
\end{Solution}
\begin{Solution}{5.{37}}
        $\Delta\lambda = \frac{\lambda}{m} = \frac{\lambda^2}{h(n-1)} = 0,5$~нм; $R = mN = 10^4$.
    
\end{Solution}
\begin{Solution}{5.{38}}
        $0,28$~нм.
    
\end{Solution}
\begin{Solution}{5.{39}}
        $31$~пм.
    
\end{Solution}
\begin{Solution}{5.{40}}
        $506$~пм.
    
\end{Solution}

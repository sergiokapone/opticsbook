% !TeX program = lualatex
% !TeX encoding = utf8
% !TeX spellcheck = uk_UA
% !TeX root =../OpticsProblems.tex

%=========================================================
\Opensolutionfile{answer}[\currfilebase/\currfilebase-Answers]
\Writetofile{answer}{\protect\section*{\nameref*{\currfilebase}}}
\chapter{Фотометрія}\label{\currfilebase}
\makeatletter
\def\input@path{{\currfilebase/}}
\makeatother
%=========================================================


%%% --------------------------------------------------------
\section{Основні поняття і закони}
%%% --------------------------------------------------------

\subsection*{Енергетичні характеристики}

Електромагнітна хвиля переносить енергію (яку будемо називати \emphz{променистою енергією}).

Основною характеристикою --- є потік випромінювання (або потужність випромінювання). \emphz{Потік випромінювання} (\hypertarget{Flux}{\emphz{променистий потік}}) --- це величина енергії, що переноситься електромагнітною хвилею за одиницю часу через ділянку:
\begin{equation}\label{eq:Flux_e}
	\Phi_e = \frac{dW_\text{е}}{dt}
\end{equation}
Потік випромінювання вимірюється у Ватах (Вт).

Енергія залежить від спектрального складу світла. Якщо розкласти хвилю на монохроматичні складові (кожна з певною довжиною хвилі), то вся енергія певним чином розподілиться між ними. Для цього вводять \emphz{спектральна густину потоку випромінювання}:
\begin{equation}\label{eq:Flux_e_lambda}
	\Phi_{e_{\lambda}} = \frac{d\Phi_e}{d\lambda}.
\end{equation}

\hypertarget{Radiant_intensity}{\emphz{Сила випромінювання}} --- характеризує потужність випромінювання в деякому напрямку і дорівнює відношенню потоку $d\Phi_e$, що поширюється від джерела всередині малого тілесного кута $d\Omega$, до величини цього кута:
\begin{equation}\label{eq:Radiant_intensity}
	I_e = \frac{d\Phi_e}{d\Omega}.
\end{equation}

Одиницею вимірювання у Міжнародній системі одиниць (СІ) є Вт/ср (Ватт на стерадіан).

\hypertarget{Radiant_exitance}{\emphz{Енергетична світність}} --- характеризує потужність випромінювання з одиниці площі поверхні і дорівнює відношенню потоку $d\Phi_e$, що випромінюється малою ділянкою поверхні джерела, до його площі $dS_S$:
\begin{equation}\label{eq:Radiant_exitance}
	M_e = \frac{d\Phi_e}{dS_S}.
\end{equation}

Одиницею вимірювання у Міжнародній системі одиниць (СІ) є Вт/м$^2$.

%%---------------------------------------------------------
%\begin{wrapfigure}{O}{0.5\linewidth}%[h!]\centering
%    \begin{tikzpicture}[
    declare function={
        angle(\dy,\dx) = atan(abs(\dy/\dx));
    },
    ]
    \contourlength{0.1pt}
    \pgfmathsetmacro\sxa{0}
    \pgfmathsetmacro\sxb{9}

    \pgfmathsetmacro\sya{-9}
    \pgfmathsetmacro\syb{9}

    %	\draw[gray!40, step=0.5] (\sxa,\sya) grid (\sxb,\syb);
    %	\draw[red,  ->] (\sxa,0) -- (\sxb,0) node[right] {$x$};
    %	\draw[red!40, ->] (0, \sya) -- (0, \syb) node[above] {$y$};
    %	\foreach \i in {\sxa,...,\sxb}
    %		{
        %    			\node[below, gray!50, font=\scriptsize] at (\i, 0) {$\i$};
        %    		}
    %	\foreach \j in {\sya,...,\syb}
    %		{
        %    			\node[left, gray!50, font=\scriptsize] at (0, \j) {$\j$};
        %    	}

    \begin{scope}[rotate=90]
        \pgfmathsetmacro\h{6}
        \coordinate (T) at (0,\h);
        \def\angle{80}
        \path[] (T) -- ++(-\angle:{\h/sin(\angle)}) coordinate (B1)  arc(0:-180: {\h/tan(\angle)} and 0.25) coordinate (B2) -- (T);
        \path[] (T) -- ++(-\angle:{1+\h/sin(\angle)}) coordinate (E1);
        \draw[] (B2) -- coordinate[pos=0.5] (C1) (E1);
        \fill[white, top color=yellow!90,bottom color=yellow!20, rotate={-23}] let \p1=(B2), \p2=(E1), \n1={{veclen(\x2-\x1,\y2-\y1)/2}} in (T) -- (E1) arc(0:-180:\n1 and 0.25 cm) -- cycle;

        \fill[white, top color=yellow!90,bottom color=yellow!20, rotate={-23}] let \p1=(B2), \p2=(E1), \n1={{veclen(\x2-\x1,\y2-\y1)/2}} in  (C1) circle (\n1 and 0.25 cm);
        \fill[themecolordark!20, opacity=0.5] (0, 0) circle({\h/tan(\angle)} and 0.25);
        \draw[dash dot] (T) -- node[left] {$r$} (C1);
        \draw (C1) -- ++({90-23}:3) coordinate (N);
    \end{scope}
    \node[above right] at (B1) {$d\Omega$};
    \node[right] at (E1) {$dS$};
%    \node[yellow!90!black, font=\large] at (T) {{\contour{blue}{\ding{90}}}};
    \shorthandoff{"}
    \pic[draw, "\alpha", angle eccentricity=1.25, angle radius=1cm] {angle=N--C1--T};
    \shorthandon{"}

\end{tikzpicture}
%    \caption{}
%    \label{pic:dSigmas}
%\end{wrapfigure}
%%---------------------------------------------------------
\hypertarget{Radiance}{\emphz{Енергетична яскравість}} --- характеризує потік випромінювання в одиниці тілесного кута з одиниці видимої площі джерела і дорівнює відношенню потоку $d\Phi_e$ з малої ділянки поверхні джерела до тілесного кута $d\Omega$ і до площі проекції цієї ділянки на площину, перпендикулярну до напряму поширення (видимої площі $d\sigma$):
\begin{equation}\label{eq:Radiance}
	B_e = \frac{d^2\Phi_e}{dS\cos\alpha \cdot d\Omega} = \frac{dI_e}{dS\cos\alpha} = \frac{dI_e}{d\sigma}.
\end{equation}


Одиницею вимірювання у Міжнародній системі одиниць (СІ) є Вт/(м$^2\cdot$ср).


\hypertarget{Irradiance}{\emphz{Опроміненість}} --- характеризує поверхневу густину потужності випромінювання, що падає на поверхню і дорівнює відношенню потоку $d\Phi_{e}$, що падає на малу ділянку поверхні, до площі цієї ділянки $dS_a$:
\begin{equation}\label{eq:Irradiance}
	E_e = \frac{d\Phi_e}{dS_a}.
\end{equation}

Одиницею вимірювання у Міжнародній системі одиниць (СІ) є Вт/(м$^2$).

\subsection*{Світлові характеристики}

Енергетичні величини вичерпно характеризують енергетичні процеси, що пов'язані з поширенням, поглинанням і перевипромінюванням електромагнітних хвиль, однак вони не дозволяють кількісно оцінити візуальне сприйняття випромінювання оком людини.

Сприйняття оком випромінювання видимого діапазону визначається не лише потужністю випромінювання, але й залежить з його спектрального складу (оскільки око --- селективний приймач випромінювання). Тому, для характеристики енергетичних властивостей, у видимому діапазоні довжин електромагнітних хвиль з урахуванням сприйняття зоровою системою ока людини  вводять так звані фотометричні (або світлові) характеристики та їх одиниці вимірювання (перелічені в в табл.~\ref{tab:photometry}).

Зв'язок світлових та енергетичних величин встановлюється через зорове сприйняття, яке добре вивчене експериментально.  Для цього вводиться функція $ V(\lambda) $ --- \emphz{відносна спектральна ефективність монохроматичного випромінювання} (або \emphz{функція видності ока}), яка пов'язує енергетичні та світлові характеристики:
\begin{equation}\label{eq:Phi_nu_with_Phi_v}
	\Phi_{v_{\lambda}} = K_m V(\lambda) \Phi_{e_{\lambda}}.
\end{equation}
Функція видності $V(\lambda)$ показує, як очі сприймають випромінювання різного спектрального складу, причому вплив потоку випромінювання з довжиною хвилі $\lambda = 555$~нм умовно приймається за одиницю, тобто $V(\lambda)$ максимальна в області жовто-зеленого кольору ($550 \div 570$~нм) і спадає до нуля для червоних та фіолетових променів (рис.~\ref{plt:LF}). Коефіцієнт $K_{m}$ --- називається \emphz{світловим еквівалентом} --- величина якого визначається використовуваною системою одиниць. У системі СІ, де за одиницю світлового потоку прийнято $1$~люмен (лм), цей коефіцієнт дорівнює $683.002$~лм/Вт.

\begin{figure}[h!]\centering
	% http://www.cvrl.org/lumindex.htm
% http://www.cvrl.org/database/text/lum/vl.htm
\begin{tikzpicture}
    \begin{axis}[trim axis left, %trim axis right,
        height=0.65\linewidth,
        width=0.9\linewidth,
        /pgf/number format/.cd,
        use comma,
        1000 sep={\,},
        xlabel = {$\lambda$, нм},
        ylabel = {$V$},
        grid=both,
        grid style={line width=.1pt, draw=blue!10},
        major grid style={line width=.2pt,draw=blue!50},
        minor tick num = 4,
        xtick distance=50,
        ytick distance=0.2,
        %        legend entries={$K_m = 683$~лм/Вт, для $\lambda = 555$~нм},
        xmin=400,
        xmax=700,
        ymin=0,
        ymax=1,
        ]
        \addplot[thick, red] table[x=l,y=V,col sep=comma, mark=none, smooth]{Photometry/LF.csv};
    \end{axis}
\end{tikzpicture}
	\caption{Крива відносної спектральної чутливості людського ока для денного зору.
        \\ (Дані взяті з \url{http://www.cvrl.org/lumindex.htm})
    }
	\label{plt:LF}
\end{figure}

Інші світлові величини визначаються аналогічно до енергетичних, однак мають відмінні назви (див. табл.~\ref{tab:photometry}) і завдяки \eqref{eq:Phi_nu_with_Phi_v} відмінні одиниці вимірювання. Наприклад, для сили світла в Міжнародній системі одиниць (СІ) вводиться одиниця, яка називається \emphz{кандела}, яка прийнята за основну. Одна кандела --- це сила світла в заданому напрямку від джерела, що випромінює монохроматичне випромінювання частотою $540\cdot10^{12}$ Гц і має силу випромінення в цьому напрямку $(1/683)$~Вт/ср.

\DefTblrTemplate{contfoot-text}{default}{\itshape\footnotesize Продовження на наступній сторінці}
\DefTblrTemplate{conthead-text}{default}{(продовження)}

\begin{small}
    \begin{longtblr}[
        caption = {Енергетичні та світлові характеристики\label{tab:photometry}},
        %    label = {tab:photometry},
        %    note{a} = {It is the first footnote.},
        %    note{$\dag$} = {It is the second long long long long long long footnote.},
        %    remark{Note} = {Some general note. Some general note. Some general note.},
        %    remark{Source} = {Made up by myself. Made up by myself. Made up by myself.},
        ]
        {
            colspec={X[l, m]X[c,m]X[l,m]X[c,m]},
            cell{1}{1,3} = {r=1,c=2}{c},
            %        cell{2}{1,3,4} = {r=2}{l},
            %        cell{8}{1,3} = {r=2}{l},
            cell{3,9}{2,4} = {c, m},
            %        cell{3}{2} = {c = 3}{c,m},
            rowhead = 1,
            hlines={white},
            vlines={white},
            row{even} = {themecolorlight!15},
            row{odd} = {themecolorlight!5},
            row{1} = {themecolordark!80, fg=white, font=\bfseries},
        }
        Енергетичні характеристики & & Світлові характеристики & \\
        % ---------------
        \hyperlink{Flux}{\bfseries Потік випромінювання}
        &
        $\Phi_e = \int\limits_0^\infty \Phi_{e_\lambda} d\lambda$, Вт
        &
        \textbf{Світловий потік}
        &
        $\Phi_v = \int\limits_0^\infty \Phi_{v_\lambda} d\lambda$, лм (люмен)
        \\
        % ---------------
        \hyperlink{Radiant_intensity}{\bfseries Сила випромінювання}
        &
        {
            $I_e = \frac{d\Phi_e}{d\Omega}$, Вт/ср
        }
        &
        \textbf{Сила світла}
        &
        {
            $I_v = \frac{d\Phi_v}{d\Omega}$,  кд (кандела)
        }
        \\
        % ---------------
        \hyperlink{Radiant_exitance}{\bfseries Енергетична світність}
        &
        $M_e = \frac{d\Phi_e}{dS}$,  Вт/м$^2$
        &
        \textbf{Світність}
        &
        $M_v = \frac{d\Phi_v}{dS}$,  лм/м$^2$
        \\
        % ---------------
        \hyperlink{Radiance}{\bfseries Енергетична яскравість}
        &
        {
            $B_e = \frac{dI_e}{d\sigma}$, Вт/(м$^2\cdot$ср)
        }
        &
        \textbf{Яскравість}
        &
        {
            $B_v = \frac{dI_v}{d\sigma}$, нт (ніт) \\[1ex]
            \hrulefill

            {
                $1\ \text{нт} = 1\ \text{кд/м$^2$}$\\
                $1\ \text{ст} = 10^4\ \text{кд/м$^2$}$ (стильб)\\
                $1\ \text{Лб} = 10^4/\pi\ \text{кд/м$^2$}$ (Ламберт)
            }
        }
        \\
        % ---------------
        \hyperlink{Irradiance}{\bfseries Опроміненість}
        &
        $E_e = \frac{d\Phi_e}{dS}$, Вт/м$^2$
        &
        \textbf{Освітленість}
        &
        {$E_v = \frac{d\Phi_v}{dS}$, \\[1ex] лк = лм/м$^2$ (люкс)
        }
        \\
    \end{longtblr}
\end{small}





%%% --------------------------------------------------------
\subsection*{Закон обернених квадратів}
%%% --------------------------------------------------------



Розглянемо точкове джерело випромінювання.

%---------------------------------------------------------
\begin{wrapfigure}[14]{O}{0.5\linewidth}%[h!]
	\centering
	\begin{tikzpicture}[
    declare function={
        angle(\dy,\dx) = atan(abs(\dy/\dx));
    },
    ]
    \contourlength{0.1pt}
    \pgfmathsetmacro\sxa{0}
    \pgfmathsetmacro\sxb{9}

    \pgfmathsetmacro\sya{-9}
    \pgfmathsetmacro\syb{9}

    %	\draw[gray!40, step=0.5] (\sxa,\sya) grid (\sxb,\syb);
    %	\draw[red,  ->] (\sxa,0) -- (\sxb,0) node[right] {$x$};
    %	\draw[red!40, ->] (0, \sya) -- (0, \syb) node[above] {$y$};
    %	\foreach \i in {\sxa,...,\sxb}
    %		{
        %    			\node[below, gray!50, font=\scriptsize] at (\i, 0) {$\i$};
        %    		}
    %	\foreach \j in {\sya,...,\syb}
    %		{
        %    			\node[left, gray!50, font=\scriptsize] at (0, \j) {$\j$};
        %    	}

    \begin{scope}[rotate=23]
        \pgfmathsetmacro\h{6}
        \coordinate (T) at (0,\h);
        \def\angle{80}
        \path[] (T) -- ++(-\angle:{\h/sin(\angle)}) coordinate (B1)  arc(0:-180: {\h/tan(\angle)} and 0.25) coordinate (B2) -- (T);
        \path[] (T) -- ++(-\angle:{1+\h/sin(\angle)}) coordinate (E1);
        \draw[] (B2) -- coordinate[pos=0.5] (C1) (E1);
        \fill[white, top color=yellow!90,bottom color=yellow!20, rotate={-23}] let \p1=(B2), \p2=(E1), \n1={{veclen(\x2-\x1,\y2-\y1)/2}} in (T) -- (E1) arc(0:-180:\n1 and 0.25 cm) -- cycle;

        \fill[white, top color=yellow!90,bottom color=yellow!20, rotate={-23}] let \p1=(B2), \p2=(E1), \n1={{veclen(\x2-\x1,\y2-\y1)/2}} in  (C1) circle (\n1 and 0.25 cm);
        \fill[themecolordark!20, opacity=0.5] (0, 0) circle({\h/tan(\angle)} and 0.25);
        \draw[dash dot] (T) -- node[left] {$r$} (C1);
        \draw (C1) -- ++({90-23}:3) coordinate (N);
    \end{scope}
    \node[above right] at (B1) {$d\Omega$};
    \node[right] at (E1) {$dS$};
    \node[yellow!90!black, font=\large] at (T) {{\contour{blue}{\ding{90}}}};
    \shorthandoff{"}
    \pic[draw, "\alpha", angle eccentricity=1.25, angle radius=1cm] {angle=N--C1--T};
    \shorthandon{"}

\end{tikzpicture}
	\caption{До пояснення закону обернених квадратів}
	\label{pic:Inverse_square_low}
\end{wrapfigure}
%---------------------------------------------------------
Точкове джерело ---- це джерело, розмірами якого можна знехтувати в порівнянні з відстанню до нього, і яке випромінює потік, рівномірний у всіх напрямках.

%З означення освітленості $E = \frac{d\Phi}{dS} $, сили світла $I = \frac{d\Phi}{d\Omega}$ та тілесного кута $d\Omega = \frac{dS\cos\alpha}{r^2}$ (рис.~\ref{pic:Inverse_square_low}) отримуємо
%\begin{equation*}
%    E = \frac{d\Phi}{dS} = \frac{Id\Omega}{dS} = \frac{I}{r^2}\cos\alpha.
%\end{equation*}

Освітленість, що створюється точковим джерелом обернено пропорційна квадрату відстані $r$ від джерела до поверхні і прямо пропорційно косинусу кута $\alpha$, між напрямком на джерело світлового потоку і нормаллю до поверхні, що освітлюється:
\begin{equation}\label{eq:Inverse_square_low}
	E =  \frac{I}{r^2}\cos\alpha.
\end{equation}
Це твердження носить назву \emphz{закону обернених квадратів}.



%%% --------------------------------------------------------
\subsection*{Закон Ламберта}
%%% --------------------------------------------------------


Джерелом випромінювання може бути не лише точкове джерело, а й деяка поверхня.


Простою моделлю протяжного джерела випромінювання є так званий \emphz{ламбертівський випромінювач}  у якого яскравість не залежить від кута спостереження. Для таких джерел сила випромінювання в даному напрямку пропорційна
косинусу кута $\alpha$ між цим напрямком і нормаллю до випромінюючої
площадки (рис.~\ref{pic:Lambert_Law}):
\begin{equation}\label{eq:Lambert_Law}
	I = I_0\cos\alpha.
\end{equation}
Цей вираз називається \emphz{законом Ламберта}.

%---------------------------------------------------------
\begin{figure}[h!]\centering%[8]{O}{0.5\linewidth}%[h!]
    \centering
    \begin{tikzpicture}
    \pgfmathsetmacro\sxa{-3} % Початкова x координата оптичної осі
    \pgfmathsetmacro\sxb{3} % Кінцева x координата оптичної осі
    \pgfmathsetmacro\sy{3}  % Координата y  оптичної осі
    \pgfmathsetmacro\sf{1}  % Координата y  оптичної осі
    \def\grid{
        % Рисування сітки
        \draw[gray!40, step=0.5] (\sxa,-\sy) grid (\sxb,\sy);
        \draw[gray!40, step=0.5] (\sxa,-\sy) grid (\sxb,\sy);
        \draw[red,  ->] (\sxa,0) -- (\sxb,0) node[right] {$x$};
        \draw[red!40, ->] (0, -\sy) -- (0, \sy) node[above] {$y$};
        \foreach \i in {\sxa,...,\sxb}
        {
            \node[below, gray!50, font=\scriptsize] at (\i, 0) {$\i$};
        }
        \foreach \j in {-\sy,...,\sy}
        {
            \node[left, gray!50, font=\scriptsize] at (0, \j) {$\j$};
        }
    }
    \draw[->] (0,0) -- (0,2.5*\sf) node[above] {$\vect{n}$};
    \fill[yellow] (\sxa, -0.125) rectangle (\sxb,0);
    \draw[domain=0:180,samples=500, blue] plot (\x:{2*\sf*(sin(\x))});
    \draw[domain=0:180,samples=500, red] plot (\x:{2*\sf});
    \foreach[] \i in {30,45, 65, 90, 105, ...,150} {
        \draw[->] (0,0) -- ++(\i:{2*\sf*(sin(\i))}) coordinate (I\i);
        %        \draw[->, red] (0,0) -- ++(\i:{4});
    }
    \node[above right] at (I90) {$I_0$};
    \node[right] at (I65) {\contour{white}{$I = I_0\cos\alpha$}};
    \node[left] at (150:{2*\sf}) {$B = B_0$};
    \draw (0,0) ++(0,1) node[above right] {$\alpha$} arc(90:65:1);
\end{tikzpicture}
    \caption{Розподіл сили світла і яскравості для плоского ламбертівського джерела світла}
    \label{pic:Lambert_Law}
\end{figure}
%---------------------------------------------------------

На практиці будь-яка поверхня, що добре розсіює, може вважатися ламбертівським випромінювачем (білий матовий папір, шорсткі поверхні металів тощо).


%%% --------------------------------------------------------
\subsection*{Освітленість від протяжних ламбертівських випромінювачів}
%%% --------------------------------------------------------


\begin{enumerate}[label*={\textcolor{themecolordark}{\ding{90}}}]

	\item Якщо поверхня освітлюється нескінченно великою рівнояскравою
	      плоскою поверхнею або увігнутою напівсферою довільного радіуса, то її освітленість дорівнює
	      \begin{equation}\label{eq:E_inf_plane}
		      E = \pi B,
	      \end{equation}
	      де $B$ --- яскравість джерела.

	\item Якщо поверхня освітлюється  рівнояскравим диском радіуса $R$ (рис.~\ref{pic:Disk_rays}), який
	      знаходиться на відстані $r$ від неї, то її освітленість дорівнює:
	      \begin{equation}\label{eq:E_disk}
		      E = \pi B \sin^2\alpha,
	      \end{equation}
	      де $2\alpha$ --- кут між крайніми променями.

	      %---------------------------------------------------------
	      \begin{center}%{O}{0.33\linewidth}%[h!]\centering
		      \begin{tikzpicture}[
    declare function={
        angle(\dy,\dx) = atan(abs(\dy/\dx));
    },
    ]
    \contourlength{0.1pt}
    \pgfmathsetmacro\sxa{0}
    \pgfmathsetmacro\sxb{9}

    \pgfmathsetmacro\sya{-9}
    \pgfmathsetmacro\syb{9}

    %	\draw[gray!40, step=0.5] (\sxa,\sya) grid (\sxb,\syb);
    %	\draw[red,  ->] (\sxa,0) -- (\sxb,0) node[right] {$x$};
    %	\draw[red!40, ->] (0, \sya) -- (0, \syb) node[above] {$y$};
    %	\foreach \i in {\sxa,...,\sxb}
    %		{
        %    			\node[below, gray!50, font=\scriptsize] at (\i, 0) {$\i$};
        %    		}
    %	\foreach \j in {\sya,...,\syb}
    %		{
        %    			\node[left, gray!50, font=\scriptsize] at (0, \j) {$\j$};
        %    	}

    \begin{scope}[rotate=180]
        \pgfmathsetmacro\h{4}
        \coordinate (T) at (0,\h);
        \coordinate (O) at (0,0);
        \def\angle{70}
        %        \draw[dashed] (-1,\h) -- (1, \h);
        \draw[themecolordark!10, fill=themecolordark!10, yshift=0.35cm, xshift=-0.34cm] (1,\h) -- ++(-45:1) -- ++(0:-2) -- ++(-45:-1) -- cycle;
        %        \node[right=2pt] at (1,\h) {\contour{white}{$S$}};
        \path[top color=yellow!80,bottom color=yellow!20] (T) -- ++(-\angle:{\h/sin(\angle)}) coordinate (B1)  arc(0:-180: {\h/tan(\angle)} and 0.25) coordinate (B2) -- (T);
        \draw[yellow, ultra thick, fill=yellow] (0, 0) circle({\h/tan(\angle)} and 0.25);
        \draw[dash dot] (T) -- (O);
        \draw[dash dot] (B2) -- (T);
        \draw (O) -- node[above] {$R$} ++(0:{-\h/tan(\angle)});
    \end{scope}
    \shorthandoff{"}
    \pic[draw, "\alpha", angle eccentricity=1.25, angle radius=2cm] {angle=B2--T--O};
    \shorthandon{"}
\end{tikzpicture}
		      \captionof{figure}{Освітлення рівнояскравим джерелом}
		      \label{pic:Disk_rays}
	      \end{center}
	      %---------------------------------------------------------


	\item Освітленість для осьової точки $A'$ зображення, яке створює об’єктив (за умови відсутності оптичних втрат), визначається як:

	      \begin{equation}\label{eq:E_objective}
		      E_{A'} = \pi B \sin^2u',
	      \end{equation}
	      $2u'$ --- задній апертурний кут об’єктива, $B$ --- яскравість об’єкта.

	      Цю формулу можна отримати з \eqref{eq:E_disk} з тих міркувань, що кут сходження крайніх променів з вершиною на поверхні  зображення замість $2\alpha$ стає рівним $2u'$.

	      %---------------------------------------------------------
	      \begin{center}%{O}{0.33\linewidth}%[h!]\centering
		      	\begin{tikzpicture}[scale=1]
    \def\sxa{-6}
    \def\sxb{4}

    \fill[line join=round, glass, draw=blue, ultra thin, name path=lens] (0.25,-2) arc (-30:30:4 and 4) -- ++(-0.5, 0) arc (150:210:4 and 4) -- cycle;

    \draw[name path=optaxis] (\sxa,0) -- (\sxb,0);

    \draw[thick, fill=black] (-0.5, 2) rectangle ++(1,0.05) (-0.5, -2) rectangle ++(1,-0.05);
    \draw[<->] (0, 2) -- node[right, pos=0.6] {$D$} ++(0, -4);

    \def\xzin{4}

    \def\d{5}
    \def\lp{0}
    \def\l{1}
    \def\f{2}
    \def\D{1.97}
    \coordinate (F) at (-\f, 0);
    \coordinate (F') at (\f, 0);
    \coordinate (A) at (-\d, 0);
    \coordinate (T) at (-\d, \l);
    \coordinate (Tp) at (-\d, \lp);
    \coordinate (B) at (-\d, -\l);
    \def\intersections{3.34}
%    \draw[thick, ->] (A) --  node[left] {$l$} (T); % об'єкт

    \foreach[count=\i] \ya in {-\D, \D} {
%        \draw[ray, yellow] (T) -- (0, \ya) coordinate (R\i); % Промінь 1
        \draw[ray, yellow!80!black] (Tp) -- (0, \ya) coordinate (R\i); % Промінь 1
%        \draw[ray, yellow] (B) -- (0, \ya) coordinate (R\i); % Промінь 1
%        \draw[ray, yellow, domain=0:\intersections, name path global=rayt\i] plot (\x, { ((\ya-\l)/\d - \ya/\f)*\x + \ya } );
        \draw[ray, yellow!80!black, domain=0:\intersections, name path global=ray\i] plot (\x, { ((\ya-\lp)/\d - \ya/\f)*\x + \ya } );
%        \draw[ray, yellow, domain=0:\intersections, name path global=rayb\i] plot (\x, { ((\ya +\l)/\d - \ya/\f)*\x + \ya } );
%        \draw[dashed, domain=0:-\d] plot (\x, { ((\ya-\lp)/\d - \ya/\f)*\x + \ya } );
    }
    \path[name intersections={of=optaxis and ray2}] (intersection-1) coordinate (Z);
    \fill[draw=yellow!90!black, thick, fill=yellow!100, opacity=0.5] (Z) circle (0.25 and  {((\D-\l)/\d - \D/\f)*\intersections + \D});
    \fill[draw=yellow!90!black, thick, fill=yellow!90, opacity=0.5]  (Tp) circle (0.25 and  \l);
%    \fill[fill=yellow, opacity=0.1]  (Tp) circle (0.5 and  5);
    %    % angles
    \shorthandoff{"}
    \pic[draw, line width=1, "$u'$", angle eccentricity=1.25, angle radius=1cm] {angle = R2--Z--F};
    \shorthandon{"}
    \draw[->, thick] (Tp) -- (T);
    \draw[->, thick] let \p1=(Z) in (Z)  -- (\x1, {((\D-\l)/\d - \D/\f)*\intersections + \D});



    \point{F'}{$F'$}{below}{red}
    \point{F}{$F$}{below}{red}
    \point{Z}{$A'$}{above right}{red}
    \point{Tp}{$A$}{above left}{red}
\end{tikzpicture}
%		      \captionof{figure}{}
%		      \label{pic:Disk_rays1}
	      \end{center}
	      %---------------------------------------------------------

	      Якщо предмет знаходиться на нескінченності, то для об'єктів з малою світлосилою
	      \begin{equation*}
		      \sin u' \approx \frac{D}{2f'},
	      \end{equation*}
	      де $D$ --- діаметр вхідної зіниці, $f'$ --- фокусна відстань об’єктива. Тоді з \eqref{eq:E_objective}  випливає, що освітленість  зображення (за умови відсутності втрат на поверхнях лінзи) дорівнює:
	      \begin{equation}\label{}
		      E_{A'} = \frac{\pi}{4B} \left( \frac{D}{f'}\right)^2.
	      \end{equation}


\end{enumerate}

%%% --------------------------------------------------------
\section{Приклади розв’язування задач}
%%% --------------------------------------------------------


\Example{На висоті $h = 3$~м над землею й на відстані $l = 4$~м від стіни висить
	лампа силою світла $I = 100$~кд. Визначити освітленість $E_1$ стіни та $E_2$
	підлоги на лінії їхнього перетину.
}

\begin{solutionexample}

	Знайдемо кути між нормаллю до поверхні в якій шукаємо освітленість і напрямком на джерело.

	%---------------------------------------------------------
	\begin{center}
		    \begin{tikzpicture}
    \contourlength{0.1pt}
    %    \pgfmathsetmacro\sxa{-5}
    %    \pgfmathsetmacro\sxb{5}
    %
    %    \pgfmathsetmacro\sya{-5}
    %    \pgfmathsetmacro\syb{5}

    %    	\draw[gray!40, step=0.5] (\sxa,\sya) grid (\sxb,\syb);
    %    	\draw[red,  ->] (\sxa,0) -- (\sxb,0) node[right] {$x$};
    %    	\draw[red!40, ->] (0, \sya) -- (0, \syb) node[above] {$y$};
    %    	\foreach \i in {\sxa,...,\sxb}
    %    		{
        %            			\node[below, gray!50, font=\scriptsize] at (\i, 0) {$\i$};
        %            		}
    %    	\foreach \j in {\sya,...,\syb}
    %    		{
        %            			\node[left, gray!50, font=\scriptsize] at (0, \j) {$\j$};
        %            	}
    \coordinate (O) at (0,0);
    \coordinate (S) at (-4.5,4);

    \path let \p1=(O), \p2=(S) in (\x1,\y2) coordinate (A1);
    \path let \p1=(O), \p2=(S) in (\x2,\y1) coordinate (A2);

    \fill[top color=yellow, bottom color=white] (S) -- ([xshift=3cm]A2) -- (O) -- cycle;
    \fill[left color=yellow, right color=white] (S) -- ([yshift=-3cm]A1) -- (O) -- cycle;

    \draw[dash dot] let \p1=(O), \p2=(S) in (S) -- node[above] {$l$} (A1);
    \draw[dash dot] let \p1=(O), \p2=(S) in (S) -- node[left] {$h$}  (A2);

    \draw[dash dot] (S) -- node[above] {$r$} (O);
    \node[text=yellow, font=\large] at (S) {\contour{blue}{\ding{90}}};
    \shorthandoff{"}
    \pic[draw, double, "$\alpha_2$", angle eccentricity=1.3, angle radius=1cm] {angle=A1--O--S};
    \pic[draw, "$\alpha_1$", angle eccentricity=1.3, angle radius=1cm] {angle=S--O--A2};

    \draw[pattern=bricks, pattern color=brown, draw=brown] let \p1=(O), \p2=(S) in (O) rectangle ++(0.5,{\y2 + 0.5cm});
    \shorthandon{"}

    \fill[pattern=north west lines] let \p1=(O), \p2=(S) in ({\x1+0.5cm}, \y1) rectangle ++({\x2-1cm}, -0.5);
    \draw[thick] let \p1=(O), \p2=(S) in ({\x1+0.5cm}, \y1) -- ++({\x2-1cm}, 0);
\end{tikzpicture}
	\end{center}
	%---------------------------------------------------------

	В кут між стіною і підлогою \emph{на стіну} випромінювання падає під кутом $\alpha_1$ (див. рис.).
	Тому, освітленість стіни згідно закону обернених квадратів \eqref{eq:Inverse_square_low}:
	\begin{equation*}
		E_1 = \frac{I}{r^2}\cos\alpha_1 = \frac{I\cdot l}{(h^2 + l^2)^{3/2}} = \frac{100\cdot4}{(4^2 + 3^3)^{3/2}} = 3,2\ \text{лк}.
	\end{equation*}

	В кут між стіною і підлогою \emph{на підлогу} випромінювання падає під кутом $\alpha_2$ (див. рис.).
	\begin{equation*}
		E_2 = \frac{I}{r^2}\cos\alpha_2 = \frac{I\cdot h}{(h^2 + l^2)^{3/2}} = \frac{100\cdot3}{(4^2 + 3^3)^{3/2}} = 2,4\ \text{лк}.
	\end{equation*}


\end{solutionexample}

%=========================================================
\Example{%
    Круглий диск радіусом $R$ світиться за законом Ламберта з яскравістю $B$. Визначити освітленість
    у точці, що віддалена на відстань  $h$ від центру диска у напрямі, нормальному до його поверхні.
    %	На скільки результат буде відрізнятись від отриманого, якщо при обчисленні освітленості приймати диск
    %	за точкове джерело, що розташоване в центрі диска з силою світла $\pi B R^2$ ?
}

\begin{solutionexample}
    %---------------------------------------------------------
    \begin{center}
        \tikzset{
    pics/.cd,
    disc/.style = {
        code = {
            \path [top color = yellow, bottom color = yellow!10]
            (0,0) ellipse [x radius = 3, y radius = 2/3];
            \path [left color = yellow, right color = yellow,
            middle color = yellow!50] (-3,0) -- (-3,-0.15) arc (180:360:3 and 2/3)
            -- (3,0) arc (360:180:3 and 2/3);
        }
    },
}
\begin{tikzpicture}

    \pgfmathsetmacro\sxa{-5}
    \pgfmathsetmacro\sxb{5}

    \pgfmathsetmacro\sya{-2}
    \pgfmathsetmacro\syb{5}

%   	\draw[gray!40, step=0.5] (\sxa,\sya) grid (\sxb,\syb);
%   	\draw[red,  ->] (\sxa,0) -- (\sxb,0) node[right] {$x$};
%   	\draw[red!40, ->] (0, \sya) -- (0, \syb) node[above] {$y$};
%   	\foreach \i in {\sxa,...,\sxb}
%   		{
%       			\node[below, gray!50, font=\scriptsize] at (\i, 0) {$\i$};
%       		}
%   	\foreach \j in {\sya,...,\syb}
%   		{
%       			\node[left, gray!50, font=\scriptsize] at (0, \j) {$\j$};
%       	}

    % Задані параметри

    \pgfmathsetmacro\H{6}
    \pgfmathsetmacro\dta{20}
    \pgfmathsetmacro\a{5}

    \pgfmathsetmacro\xt{0}

    % Параметр вздовж висоти
    \pgfmathsetmacro\l{0.5}

    % Радіус кола для тілесного кута

    \pgfmathsetmacro\r{\l*\H*sin(\a)/cos(\dta - \a)}

    % Радіус основи

    \pgfmathsetmacro\Ro{\H*sin(\a)/cos(\dta - \a)}

    % Радіус плями

    \pgfmathsetmacro\R{\Ro*cos(\a)/cos(\dta + \a)}

    % Координата центра плями

    \pgfmathsetmacro\xC{\H*tan(\dta - \a) + \xt + \R}

    % ------ Побудови -------

    % --- Координати ---

    % Вершина конуса
    \coordinate (T) at (\xt, \H);

    % Точка проекції основи конуса на площину
    \coordinate (B) at ({\H*tan(\dta) + \xt}, 0);

    % Нога 1

    \coordinate (B1) at ({\H*tan(\dta - \a) + \xt}, 0);

    % Нога 2

    \coordinate (B2) at ({\H*tan(\dta + \a) + \xt}, 0);

    % Точка кола на конусі

    \coordinate (C) at ({\l*\H*tan(\dta) + \xt}, {(1 - \l)*\H});

    % Координата центра плями

    \coordinate (C1) at (\xC, 0);

    % Побудова конуса
    \fill[top color = yellow!50, bottom color=yellow!10] (T) -- (B2) arc(0:-180:{\R} and 0.25) -- cycle;
    \fill[yellow!25, opacity=0.5] (C1)  circle({\R} and 0.25) ;
    \draw[dashed, opacity=0.5] (T) --node[left=0.25 cm] {$r$}  (B);


    \coordinate (D) at (\xC, \H); % Координата центра диска
    \path[opacity=0.5] (D) pic {disc} ;
    \draw[dash dot] (C1) -- node[right] {$h$} (D);

    % Побудова кільця
    \pgfmathsetmacro\d{0.5} % Товщина кільця
    \pgfmathsetmacro\ph{3} % Кут d\phi

    \fill[yellow, opacity=0.5, even odd rule, name path=circles] let \p1=(D) in (D) circle[x radius={\x1 - \d cm /2}, y radius={\d*0.4 cm}] circle[x radius={\x1 + \d cm /2}, y radius={\d*0.6 cm}];

    \path[name path=line1, thin] (D)--  ++(\ph:-5);
    \path[name path=line2, thin] (D) -- ++(-\ph:-5);
    \draw[name intersections={of=circles and line1}] (D) --  node[above=0.2cm] {$d\phi$}  (intersection-2) ;
    \draw[name intersections={of=circles and line2}] (D) -- (intersection-2);

    \begin{scope}
        \clip  (D) -- +(-\ph:-5) -- +(\ph:-5) -- cycle;
        \fill[pattern=crosshatch, opacity=0.5, even odd rule] let \p1=(D) in (D) circle[x radius={\x1 - \d cm /2}, y radius={\d*0.4 cm} ] circle[x radius={\x1 + \d cm /2}, y radius={\d*0.6 cm}];
    \end{scope}

    \draw let \p1=(D) in (C1) -- ({\xt -\d/2}, \y1) coordinate (theta1);
    \draw let \p1=(D) in (C1) -- ({\xt + \d/2}, \y1) coordinate (theta2);

    \begin{pgfonlayer}{bg}
        \node[trapezium, trapezium left angle={180-30}, trapezium right angle=30, minimum size=2.5cm, shade,shading=axis,shading angle=120, inner color=yellow!30, outer color=white, draw] at (C1) {};
    \end{pgfonlayer}

    \draw (D) --node[above] {$R$} ++(0:3) coordinate (E);
    \draw[dash dot] (C1) -- (E);
%    \draw[dash dot] (theta1) -  ++(-180:1.5) coordinate (N1);
%    \draw[dash dot] let \p1=(T), \p2=(D) in (theta2) --  ++({atan(veclen(\y2 - \y1,\x2 - \x1)/\H cm) + 180 -\a/2}:2) coordinate (N2);

    \point{C1}{$A$}{below=0.25cm}{}

    \node (dS) at ([shift={(-1.125,1)}]T) {
        $\underset{\text{елемент поверхні диска}}{dS = \frac{d\sigma}{\cos\theta}}$
    };

    \draw[-to] (dS) to [out=-90, in=90] (T);

    \shorthandoff{"}
    \pic[ draw,"\theta", angle eccentricity=1.2, angle radius=2.6cm] {angle=D--C1--theta2};
    \pic[fill=gray!50, opacity=0.5, draw, "d\theta", angle eccentricity=1, angle radius=2.5cm, pic text options={shift={(-0.5cm,0)}}] {angle=theta2--C1--theta1};

        \pic[draw, "$\theta_0$", angle eccentricity=1.25, angle radius=1.5cm] {angle=E--C1--D};
    \shorthandon{"}


\end{tikzpicture}
    \end{center}
    %---------------------------------------------------------

    Весь диск розріжемо на нескінченно малі світні точки, для яких освітленість в точці $A$ визначається за законом обернених квадратів \eqref{eq:Inverse_square_low}:
    \begin{equation}\label{eq:dE}
        dE = \frac{dI}{r^2}\cos\theta.
    \end{equation}

    З означення яскравості
    \begin{equation}\label{eq:dI}
        dI = BdS\cos\theta = Bd\sigma,
    \end{equation}
    де $dS$ ---  елемента площі поверхні диска, $d\sigma$ --- видима з точки $A$  площа диска,
    яку можна знайти з означення тілесного кута~\eqref{eq:solid_angle}):
    \begin{equation}\label{eq:dsigma}
        d\sigma = r^2 d\Omega = r^2 \sin\theta d\theta d\phi.
    \end{equation}


    Таким чином, враховуючи \eqref{eq:dI} та \eqref{eq:dsigma}, вираз \eqref{eq:dE} приймає вигляд:
    \begin{equation*}
        dE = \frac{Bd\sigma}{r^2}\cos\theta = B r^2 \cos\theta\sin\theta d\theta d\phi.
    \end{equation*}
    Проінтегруємо цей вираз по площі всього диска, враховуючи що диск є ламбертівським джерелом ($B = \mathrm{const}$):
    \begin{equation}\label{eq_ex:E}
        E =  B \int\limits_0^{2\pi} d\phi \int\limits_0^{\theta_0}  r^2 \cos\theta\sin\theta d\theta = \pi B \sin^2\theta_0,
    \end{equation}
    де $2\theta_0$ --- кут між крайніми променями від  диска в точці $A$.

    З умов задачі $\sin\theta_0 = \frac{R}{\sqrt{R^2 + h^2}}$, тому з \eqref{eq_ex:E} остаточно маємо:
    \begin{equation}
        E = \frac{B \pi  R^2}{\sqrt{R^2 + h^2}}.
    \end{equation}

    \emph{Рефлексія після розв'язку}:


    \begin{enumerate}
        \item  Величина $B \cdot \pi R^2 = BS_\text{диска} = I$ --- є силою світла диска, який умовно сконцентровано в точку в його центрі.

        \item  Якщо диск має нескінченний радіус ($\theta_0 \to \pi/2$), то формула \eqref{eq_ex:E} дає вираз
        \begin{equation*}
            E = \pi B,
        \end{equation*}
    \end{enumerate}
    що співпадає з освітленістю від нескінченної рівнояскравої площини \eqref{eq:E_inf_plane}.


\end{solutionexample}

%=========================================================

\Example{Люмінесцентна циліндрична лампа діаметром $d = 2,5$~см і
	довжиною $l = 40$~см створює на відстані $r = 5$~м у напрямку,
	перпендикулярному осі лампи, освітленість $E_v = 2$~лк. Приймаючи лампу
	за косинусний випромінювач, визначити: 1) силу світла $I$ у даному
	напрямку; 2) яскравість $L$; 3) світність $M$ лампи.}

\begin{solutionexample}
	Більший із двох розмірів лампи --- довжина --- в $12$ разів менше відстані,
	на якій виміряна освітленість. Отже, для обчислення сили світла в
	даному напрямку можна прийняти лампу за точкове джерело та
	застосувати формулу
	\begin{equation*}
		E = \frac{I}{r^2},
	\end{equation*}
	звідки $I = 25$~кд.
	Для обчислення яскравості застосуємо формулу
	\begin{equation*}
		B = \frac{I}{\sigma}
	\end{equation*}
	де $\sigma$ --- площа проекції протяжного джерела світла на площину,
	перпендикулярну напрямку спостереження.

	У випадку циліндричної люмінесцентної лампи проекція має форму
	прямокутника довжиною $l$ і шириною $d$. Отже, знайдемо $B = 2,5\cdot10^3$~кд/м$^2$.

	Так як люмінесцентну лампу можна вважати косинусним
	випромінювачем, то її світність $M = \pi B = 7,9\cdot10^3$~лк.
\end{solutionexample}

\Example{Прожектор ближньої дії дає пучок світла у вигляді конуса з кутом
	розкриву $2\theta = 40^\circ$. Світловий потік $\Phi$ прожектора дорівнює $8\cdot10^4$~лм. Припускаючи, що світловий потік всередині конуса розподілений
	рівномірно, визначити силу світла прожектора.}

\begin{solutionexample}

	Повний тілесний кут прожектора можна визначити в сферичній системі координат з \eqref{eq:solid_angle_in_spherical}:
	\begin{equation*}
		\Omega = \int\limits_0^{\theta_0} \int\limits_0^{2\pi} \sin\theta d\theta d\phi = 2\pi (1- \cos\theta_0).
	\end{equation*}
	З означення сили світла $I = \frac{\Phi}{\Omega}$ маємо
	\begin{equation*}
		I = \frac{\Phi}{2\pi (1- \cos\theta_0)} = \frac{8\cdot10^{-4}}{2\pi(1- 0,94)} = 2,11\cdot10^5\ \text{кд}.
	\end{equation*}
\end{solutionexample}


%%% --------------------------------------------------------
\section{Задачі для самостійного розв’язку}
%%% --------------------------------------------------------


%=========================================================
\begin{problem}%2.1
Як встановлюється зв’зок між люменом і ватом?
\end{problem}


%=========================================================
\begin{problem}%2.2
Як використовується закон обернених квадратів для визначення сили
світла методом суб’єктивної фотометрії?
\end{problem}


%=========================================================
\begin{problem}%2.3
Наведіть приклади ламбертових випромінювачів.
\end{problem}


%=========================================================
\begin{problem}%2.4
Сформулюйте інваріант Штраубеля
\end{problem}


%=========================================================
\begin{problem}%2.5
Оцінити похибку наближення освітленості, яка створюється
рівнояскравим диском деякого розміру, до освітленості від точкового
джерела.
\end{problem}






%%% --------------------------------------------------------
\subsection*{Світловий потік і сила світла}
%%% --------------------------------------------------------


%=========================================================
\begin{problem}%2.6
Лампочка, яка споживає потужність $P  = 75$~Вт, створює на відстані $r
	= З$~м при нормальному падінні променів освітленість $E = 8$~лк.
Визначити питому потужність $p$ лампочки (у ватах на канделу) і
світлову віддачу $\eta$ лампочки (у лм/Вт).
\begin{solution}
	$1$~Вт/кд;  $12,1$~лм/Вт.
\end{solution}
\end{problem}


%=========================================================
\begin{problem}%2.7
Тепловий фотоприймач (див. рис.) --- це камера з
площею внутрішньої поверхні $S = 2$~см$^2$, яка має
невеликий отвір площею $S_1 =1$~мм$^2$.
%\leavevmode{
\InsertBoxR{0}{%
    \parbox{0.25\textwidth}{
    \begin{center}
        % BLACK BODY
\begin{tikzpicture}[
   	scale=1.5,
	radiation/.style={
	-{Latex[length=2,width=1.5]},red!95!black!50,opacity=0.7,very thin,decorate,
	decoration={snake,amplitude=0.7,segment length=2,post length=2}
	},
	rotate=10]

	\shade[top color=black!60,bottom color=black!80,shading angle=10] % background
	(7:1) arc (7:355:1);

	\fill[thick,black,postaction=decorate, % rough inner surface
		decoration={markings,mark=between positions 0.55 and 1 step 0.03 with {
						\node[transform shape,inner sep=1pt]
						(hit\pgfkeysvalueof{/pgf/decoration/mark info/sequence number}) {};
					}}]
	(7:1) arc (7:353:1) --++ (-7:-0.18)
	decorate[decoration={random steps,segment length=2,amplitude=1pt}]
		{arc (-7:-353:0.82)} -- cycle;

	\draw[yellow] % connect light ray to random points
	(8:1.5) -- (hit6.center) -- (hit1.center) -- (hit15.center) -- (hit5.center) --
	(hit9.center) -- (hit14.center) -- (hit2.center) -- (hit10.center) -- (hit3.center) --
	(hit4.center) -- (hit11.center) -- (hit13.center);

	\foreach \ang in {-35,-5,35}{
			\draw[radiation] (1,0)++(\ang:0.1 and 0.2) --++ (\ang:0.35);
		}

\end{tikzpicture}
    \end{center}
    }
}[1]

Внутрішня поверхня поглинає незначну кількість світла
(коефіцієнт поглинання $k_\text{п} = 0,01$), а іншу частину розсіює. В цих
умовах всередені фотоприймача утворюється рівномірно розподілене за
всіма напрямками випромінювання. Яка частина світлового потоку
$\nfrac{\Phi}{\Phi_0}$ (де $\Phi_0$ --- потік, який падає на вхідний отвір камери) виходить
через отвір назад?

%%---------------------------------------------------------
%\begin{center}
%	% BLACK BODY
\begin{tikzpicture}[
   	scale=1.5,
	radiation/.style={
	-{Latex[length=2,width=1.5]},red!95!black!50,opacity=0.7,very thin,decorate,
	decoration={snake,amplitude=0.7,segment length=2,post length=2}
	},
	rotate=10]

	\shade[top color=black!60,bottom color=black!80,shading angle=10] % background
	(7:1) arc (7:355:1);

	\fill[thick,black,postaction=decorate, % rough inner surface
		decoration={markings,mark=between positions 0.55 and 1 step 0.03 with {
						\node[transform shape,inner sep=1pt]
						(hit\pgfkeysvalueof{/pgf/decoration/mark info/sequence number}) {};
					}}]
	(7:1) arc (7:353:1) --++ (-7:-0.18)
	decorate[decoration={random steps,segment length=2,amplitude=1pt}]
		{arc (-7:-353:0.82)} -- cycle;

	\draw[yellow] % connect light ray to random points
	(8:1.5) -- (hit6.center) -- (hit1.center) -- (hit15.center) -- (hit5.center) --
	(hit9.center) -- (hit14.center) -- (hit2.center) -- (hit10.center) -- (hit3.center) --
	(hit4.center) -- (hit11.center) -- (hit13.center);

	\foreach \ang in {-35,-5,35}{
			\draw[radiation] (1,0)++(\ang:0.1 and 0.2) --++ (\ang:0.35);
		}

\end{tikzpicture}
%\end{center}
%%---------------------------------------------------------


\begin{solution}
	$\frac{\Phi}{\Phi_0} = \frac{1}{1 + k_\text{п}\frac{S - S_1}{S_1}}$.
\end{solution}
\end{problem}


%=========================================================
\begin{problem}%2.8
Яку силу струму $I$ покаже гальванометр, приєднаний до селенового
фотоелемента, якщо на відстані $r = 75$~см від нього помісити лампочку,
повний світловий потік якої дорівнює $1,2\cdot 10^3$ лм? Площа робочої
поверхні фотоелемента рівна $10$~см$^2$, чутливість $S_i = 3\cdot 10{-4}$~А/лм.
\begin{solution}
	$51$~мкА.
\end{solution}
\end{problem}


%=========================================================
\begin{problem}%2.9
Плоско-опуклу лінзу розрізали навпіл і склали плоскими боками дві
половинки. Як зміниться яскравість зображення віддаленого предмета?
\begin{solution}
	Не зміниться.
\end{solution}
\end{problem}





%%% --------------------------------------------------------
\subsection*{Освітленість}
%%% --------------------------------------------------------

%=========================================================
\begin{problem}%2.10
При друкуванні фотознімка негатив освітлювався протягом $t_1 = 3$~с
лампочкою силою світла $I_1 = 15$~кд з відстані $r_1 = 50$~см. Визначити час
$t_2$, протягом якого потрібно освітлювати негатив лампочкою з силою
світла $I_2 = 60$~кд з відстані $r_2 = 2$~м, щоб одержати відбиток з таким ж
ступенем почорніння, як і в першому випадку?
\begin{solution}
	$12$~с.
\end{solution}
\end{problem}


%=========================================================
\begin{problem}%2.11
На висоті $h = 3$~м над землею й на відстані $r = 4$~м від стіни висить
лампа силою світла $I = 100$~кд. Визначити освітленість $E_1$ стіни й $E_2$
горизонтальної поверхні землі на лінії їхнього перетину.
\begin{solution}
	$3,2$~лк; $2,4$~лк.
\end{solution}
\end{problem}


%=========================================================
\begin{problem}%2.12
На щоглі висотою $h = 8$~м висить лампа силою світла $I = 10^3$~кд.
Приймаючи лампу за точкове джерело світла, визначити, на якій
відстані $l$ від основи щогли освітленість $E$ поверхні землі дорівнює $1$ лк.
\begin{solution}
	$18,3$~м.
\end{solution}
\end{problem}


%=========================================================
\begin{problem}%2.13
На якій висоті $h$ над центром круглого стола радіусом $r = 1$~м
потрібно повісити лампочку, щоб освітленість на краю стола була
максимальною?
\begin{solution}
	$0,707$~м.
\end{solution}
\end{problem}




%%% --------------------------------------------------------
\subsection*{Яскравість і світність}
%%% --------------------------------------------------------


%=========================================================
\begin{problem}%2.14
Світильник з матового скла має форму кулі діаметром $d = 20$~см. Сила
світла $I$ кулі дорівнює $80$~кд. Визначити повний світловий потік $\Phi$,
світність $M$ и яскравість $B$ світильника.
\begin{solution}
	$2$~клм; $8$~клк; $2,5$~ккд/м$^2$.
\end{solution}
\end{problem}

%=========================================================
\begin{problem}%2.15
Якою виявиться освітленість $E$ ділянки, якщо джерелом світла є
нескінченна площина, паралельна цій ділянці? Поверхнева яскравість
джерела світла всюди однакова і не залежить від напрямку.
\begin{solution}
	$E = \pi B$.
\end{solution}
\end{problem}


%=========================================================
\begin{problem}%2.16
Знайти освітленість на горизонтальній ділянці, яка освітлюється
небесною напівсферою, за умови рівномірної яскравості неба.
\begin{solution}
	$E = \pi B$.
\end{solution}
\end{problem}


%=========================================================
\begin{problem}%2.17
Сонце, перебуваючи поблизу зеніту, створює на горизонтальній
поверхні освітленість $E = 10^5$~лк. Діаметр Сонця видний під кутом $\alpha =
	32'$. Визначити видиму яскравість $B$ Сонця.
\begin{solution}
	$B = \frac{E}{\pi\sin^2\frac{\alpha}{2}} = 1,5$~Гкд/м$^2$.
\end{solution}
\end{problem}


%=========================================================
\begin{problem}%2.18
Яскравість $B$ куба, що світиться однакова у всіх напрямках і дорівнює
$5\cdot10^3$~кд/м$^2$. Ребро а куба рівне $20$~см. В якому напрямку сила світла $I$
куба максимальна? Визначити максимальну силу світла $I_{\max}$ куба.
\begin{solution}
	по діагоналі куба; $I_{\max} = \sqrt3 Ba^2 = 350$~кд.
\end{solution}
\end{problem}


%=========================================================
\begin{problem}%2.19
Яскравість лампи з плафоном, який має форму конуса висотою $15$~см з
діаметром $20$~см, однакова в усіх напрямках і дорівнює $2 \cdot 10^3$~кд/м$^2$.
Основа конуса не світиться. Визначити силу світла конуса в напрямках:
1) вздовж осі; 2) перпендикулярно до осі.
\begin{solution}
	1) $63$~кд; 2) $30$~кд.
\end{solution}
\end{problem}


%=========================================================
\begin{problem}\label{prb:2.20}%2.20
На висоті $h = 1$~м над горизонтальною площиною паралельно їй
розташований невеликий диск, що світиться. Сила світла $I_0$ диска в
напрямку його осі дорівнює $100$~кд. Приймаючи диск за точкове
джерело з косинусним розподілом сили світла, знайти освітленість $E$
горизонтальної площини в точці $A$, віддаленої на відстань $r = 3$~м від
точки, розташованої під центром диска.
\begin{solution}
	$E = \frac{I_0h^2}{(h^2 + r^2)^2} = 1$~лк.
\end{solution}
\end{problem}


%=========================================================
\begin{problem}%2.21
На якій висоті $h$ над горизонтальною площиною (див. попереднє
завдання~\ref{prb:2.20}) потрібно помістити диск, що світиться, щоб освітленість у
точці $A$ була максимальною?
\begin{solution}
	$3$~м.
\end{solution}
\end{problem}


%=========================================================
\begin{problem}%2.22
Визначити освітленість $E$, світність $M$ і яскравість $B$ кіноекрана, який
рівномірно розсіює світло у всіх напрямках, якщо світловий потік $\Phi$, що
падає на екран з об'єктива кіноапарата (без кінострічки), дорівнює
$1,75\cdot10^3$~лм. Розмір екрану $5 \times 3,6$~м, коефіцієнт відбиття $\rho = 0,75$.
\begin{solution}
	$97$~лк; $73$~лк; $23$~кд/м$^2$.
\end{solution}
\end{problem}


%=========================================================
\begin{problem}%2.23
Освітленість $E$ поверхні, покритої шаром сажі, дорівнює $150$~лк,
яскравість $B$ однакова у всіх напрямках і дорівнює $1$~кд/м$^2$. Визначити
коефіцієнт відбиття $\rho$ сажі.
\begin{solution}
	$0,02$.
\end{solution}
\end{problem}


\Closesolutionfile{answer}


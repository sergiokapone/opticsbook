% !TeX program = lualatex
% !TeX encoding = utf8
% !TeX spellcheck = uk_UA

\clearpage
\section{Фотометричні дані}

\begin{table}[!ht]
    \caption{Сили світла деяких джерел}
    \centering
    \begin{tblr}
        {
            colspec={X[l,m]Q[c,m]},
            row{1} = {c, font=\bfseries}
        }
        \toprule
        Джерело                                          & Сила світла, кд  \\ \midrule
        Сонце                                            & $3\cdot 10^{27}$ \\
        Повний місяць                                    & $1,4\cdot 10^{8}$ \\
        Портативний лазер                                & $5 - 500$        \\
        Морський маяк                                    & {$10^5 - 10^7$}  \\
        Лампи розжарення                                 & $5 - 500$        \\
        Свічка                                           & $0.5 - 2$        \\  \bottomrule
    \end{tblr}
\end{table}

%\clearpage
\begin{table}[!ht]
    \caption{Освітленості від деяких джерел}
    \centering
    \begin{tblr}{
            colspec={X[l,m]Q[c,m]},
            row{1} = {c, font=\bfseries}
        }
        \toprule
        Джерело                                          & Освітленість, лк \\ \midrule
        Світло Сіріуса, найяскравішої зірки нічного неба & $10^{−5}$        \\
        Поверхня Землі під прямими сонячними променями & $32000 - 10000$        \\
        Безмісячне зоряне небо зі світінням              & $0.002$          \\
        Чверть Місяця                                    & $0.01$           \\
        Повний місяць у ясну ніч                         & {$0.27 - 1.0$}   \\
        Повний місяць в тропіках                         & $1$              \\
        У морі на глибині $\approx 50$ м                 & до $20$          \\
        Житлова кімната                                  & $50$             \\
        Дуже темний похмурий день                        & $100$            \\
        Офісне освітлення                                & $320 - 500$      \\ \bottomrule
    \end{tblr}
\end{table}

\clearpage
\begin{table}[!ht]
    \caption{Яскравість деяких випромінювачів}
    \centering
    \begin{tblr}{
            colspec={X[l,m]Q[c,m]},
            row{1} = {c, font=\bfseries}
        }
        \toprule
        Джерело                                          & Яскравість, кд/м$^2$ \\ \midrule
        Сонце                                      & $1,5 - 10^{8}$        \\
        Нічне небо за відсутності Місяця           & $10$        \\
        Вольфрамова нитка лампи розжарювання       & $(1,5- 2)\cdot 10^6$          \\
        Поверхня Місяця                            & $2.5 \cdot 10^3$           \\
        Екран кінотеатру                        & {$5 - 20$}   \\
        Аркуш білого паперу за освітленості $ 30 - 50 $~лк                         & $10 - 15$              \\
        сніг під прямими сонячними променями                 & $ 3 \cdot 10^4 $          \\ \bottomrule
    \end{tblr}
\end{table}
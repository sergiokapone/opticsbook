\protect \section *{\nameref *{Interference}}
\begin{Solution}{4.{8}}
	а) $ 1,2087 $; $ 0,227 $~мкм; б) $ 93,1 $ \%; $ 96,49 $ \%; в) $ 3,39 $ \%.
\end{Solution}
\begin{Solution}{4.{9}}
    $ d =  3,1$~мкм, $ R=0,978 $.
\end{Solution}
\begin{Solution}{4.{10}}
    Показник заломлення матеріалу покриття $ n_1 = \sqrt{n_2} = 1,265  $.
    Мінімальна товщина одношарового покриття $ d_{\min} =  \frac{1}{n_1}\frac\lambda4 = 0,11$~мкм.
    Коефіцієнт проходження світла без покриття:
    \begin{equation*}
        T_I = (1 - R_{02})^2 = \left[ 1 - \left( \frac{n_0 - n_2}{n_0 + n_2}\right)^2 \right]^2 \approx 0.896.
    \end{equation*}
     Коефіцієнт проходження світла з покриттям:
     \begin{equation*}
         T'_I = (1 - R_{01})^2 (1 - R_{12})^2 = \left[ 1 - \left( \frac{n_0 - n_1}{n_0 + n_1}\right)^2 \right]^2 \left[ 1 - \left( \frac{n_1 - n_2}{n_1 + n_2}\right)^2 \right]^2 \approx 0,944.
     \end{equation*}

     Втрати на відбиття складають відповідно:
     \begin{equation*}
         R_I = 1 - T_I = 0,104, \quad R'_I = 1- T'_I = 0,054,
     \end{equation*}
     тобто зменшуються майже удвічі.
\end{Solution}
\begin{Solution}{4.{11}}
	$ E = 5 $~В/м; $\omega = \pi\cdot10^{15}$~рад/с; $\phi_0 = 0,205\pi$; $ E = 5\sin(\pi(10^{15} t + 0.205)) $.
\end{Solution}
\begin{Solution}{4.{12}}
	\emph{Підказки}.
	\begin{enumerate}
		\item  Скористатися комплексною формою подання $E_m$ при $x = 0$.
		\item  Скористатися формулами для суми членів скінченної (а) і нескінченної (б) спадаючих
		      геометричних прогресій.
		\item  Визначити $ E_{0_2} $ множенням комплексної амплітуди на спряжений їй вираз й
		      скористатися формулою Ейлера.
	\end{enumerate}
	а) $ E_0 = E_{0_1} \frac{\sin\frac{N\delta}{2}}{\sin\frac{\delta}{2}} $, $ I = I_1 \left( \frac{\sin\frac{N\delta}{2}}{\sin\frac{\delta}{2}}\right)^2 $; б) $ E_0 = E_{0_1} \frac1{2\sin\frac{\delta}{2}} $, $ I = I_1 \frac1{4\sin^2\frac{\delta}{2}} $.
\end{Solution}
\begin{Solution}{4.{13}}
	$ 500 $ нм.
\end{Solution}
\begin{Solution}{4.{14}}
	$ 1,000865 $.
\end{Solution}
\begin{Solution}{4.{15}}
	а) $ 0,13 $ мкм; $ 0,09 $ мкм; б) не зміниться.
\end{Solution}
\begin{Solution}{4.{16}}
	а) $ 0,24 $ мкм; $ 0,12 $ мкм; б) $ 0,12 $ мкм; $ 0,24 $ мкм.
\end{Solution}
\begin{Solution}{4.{17}}
	а) $ 589 $ нм; б) $ 673 $ нм.
\end{Solution}
\begin{Solution}{4.{18}}
	$ 3,54 $ мм.
\end{Solution}
\begin{Solution}{4.{19}}
	$ 1,56 $.
\end{Solution}
\begin{Solution}{4.{20}}
	$ 1,00038 $.
\end{Solution}
\begin{Solution}{4.{21}}
	а) $r_m^{\text{світлого}} = \sqrt{\frac{R_1R_2}{n_2(R_1 - R_2)} (2m-1)\frac\lambda2 }$; б)   $r_m^{\text{темного}} = \sqrt{\frac{R_1R_2}{n_2(R_1 - R_2)} m\lambda}$.
\end{Solution}
\begin{Solution}{4.{23}}
	$ 0,541 $ мкм.
\end{Solution}
\begin{Solution}{4.{24}}
	$ 1,5 $.
\end{Solution}
\begin{Solution}{4.{25}}
	$\frac{\Delta\lambda}{\lambda} = \frac1{980}$; $\Delta\lambda = 6,02$~\AA.
\end{Solution}
\begin{Solution}{4.{26}}
	$N = \frac{dD^2}{35nf^2\lambda} = 2$.
\end{Solution}
\begin{Solution}{4.{27}}
	$D_{\min} = \frac{36nf^2\lambda}{D^2} = 0,81$~мм.
\end{Solution}
\begin{Solution}{4.{28}}
	$a = \frac{f\lambda}{ \Delta x} = 0,06$~мм.
\end{Solution}
\begin{Solution}{4.{29}}
	$ \frac{\Delta l}{l}  = \frac{2ha\sqrt{n^2 - 1}}{\lambda} = 0,75$.
\end{Solution}
\begin{Solution}{4.{30}}
	Видимим є лише нульовий порядок: спочатку при відбитті від передньої грані
	при $ \Delta x = 0 $, потім від задньої грані при $ \Delta  x = 2dn = 2 $~ см. Наступні порядки
	інтерференції --- спадаючої інтенсивності.
\end{Solution}
\begin{Solution}{4.{31}}
	$V = \frac{\sin\frac{\pi b d}{\lambda l}}{\frac{\pi b d}{\lambda l}}$.
\end{Solution}
\begin{Solution}{4.{32}}
	$D \approx \frac{\lambda}{\alpha} = 0,05$~мм.
\end{Solution}
\begin{Solution}{4.{33}}
	$ l \frac{fD\alpha}{\lambda} \approx 100$~см.
\end{Solution}
\begin{Solution}{4.{34}}
	$m = \frac{\Delta}{\lambda} \approx 360$ (де $\Delta = 2h\sqrt{n^2 - \sin^2\phi}$ --- різниця ходу променів), кут сходження променів в силу симетрії приблизно дорівнює апертурі інтерференції:
	\begin{equation*}
		\Omega = \frac{L}{h} \tg\psi\cos^2\phi = 2\cdot10^{-4},
	\end{equation*}
	де $\psi$ --- кут заломлення, ширина інтерференційних смуг, $\Delta \approx \frac{\lambda}{\omega} = 2,8$~см, граничний розмір джерела $b \approx \Delta \approx 2,8$~см. Допустима немонохроматичність
	$ \Delta\lambda =  \frac{\lambda}{m} = 1,6 $~нм.
\end{Solution}
\begin{Solution}{4.{35}}
	$ m_{\max} \approx 1000$, $ m_{\min} \approx 720 $. $ \Delta\lambda \le 0,5 $~нм. Джерело світла може мати будь-які розміри.
\end{Solution}
\begin{Solution}{4.{36}}
	1) $\Delta x \approx 10^{-3}$~см; 2) $|x| \le 0,25$~см (область локалізації смуг); 3) $ m_{\max} = 250$, $ m_{\min} = 0$, $N \approx 500$, 4) $\Delta\lambda = 20$~\AA; 5) $ b \le 10^{-3}$~см.
\end{Solution}
\begin{Solution}{4.{37}}
	1) $L = \frac{D}{4\theta (n - 1)} = 1$~м; 2) $ \Delta\lambda = \frac{\lambda}{m_{\max}} = 5$~нм, де $m_{\max} = 100$; 3) $\psi \le \frac{2\lambda}{D} = 5\cdot10^{-5} = 0.18''$.
\end{Solution}
\begin{Solution}{4.{38}}
	1) $L = \frac{D - a}{2\alpha}$, де $\alpha = \frac{a}{f} = 10^{-2}$, $ \Delta x  = \frac{\lambda}{\alpha} = 5\cdot10^{-3}$~см, $N = 200$; 2) $m_{\max} = 100$,  $ \Delta\lambda = \frac{\lambda}{m_{\max}} = 5$~нм; 3) $b \le \frac{2\lambda f}{D - a} = 2,5\cdot10^{-3}$~см.
\end{Solution}
\begin{Solution}{4.{39}}
	$V(L) = \left| \frac{\sin(0,1\pi L)}{0,1\pi L}\right|$, смуги розмиваються при $L = 10m$~см, де $m = 1, 2, 3, \ldots$.
\end{Solution}
\begin{Solution}{4.{40}}
	а) В центрі кілець максимальний порядок інтерференції. Із збільшенням номеру
	$k$ порядок їх інтерференції $ m $ зменшується $\frac{dk}{dm} < 0$. б) кутова відстань між кільцями зменшується із зменшенням порядку інтерференції  $\frac{\Delta \epsilon'_2}{dm} > 0 $.
\end{Solution}

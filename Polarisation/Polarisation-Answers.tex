\protect \section *{\nameref *{Polarisation}}
\begin{Solution}{6.{15}}
      а) $ R = 0,0502 $; $ P_R = 0,8308 $; $ P_D = 0,0439 $; б) $ 6,25 \% $; $ 76,96 \% $; $ 5,13 \% $; в) $ 8,32 \% $; $ 74,65 \% $; $ 6,78 \% $.
    
\end{Solution}
\begin{Solution}{6.{16}}
        $ 36^\circ56'20'' $; $ 4,01 \% $.
    
\end{Solution}
\begin{Solution}{6.{17}}
        а) $ \epsilon_\text{Б} = 56^\circ 18' 38'' $; $ R = 7,39 \% $; $ P_D = 7,98 \% $; б) $ \epsilon_\text{Б} = 58^\circ 18' 49'' $; $ R =  10,04 \%$; $ R_D = 11,16 \% $; в) $ \epsilon_\text{Б} = 59^\circ 58' 14'' $; $ R =  12,45 \%$; $ R_D = 14,23 \% $.
    
\end{Solution}
\begin{Solution}{6.{18}}
        $ 61^\circ12' $.
    
\end{Solution}
\begin{Solution}{6.{19}}
        $ 32^\circ $.
    
\end{Solution}
\begin{Solution}{6.{20}}
        $ 1,52 $.
    
\end{Solution}
\begin{Solution}{6.{21}}
       а) $ 106^\circ $; б) $ 156^\circ $; в) $ 100^\circ $.
    
\end{Solution}
\begin{Solution}{6.{22}}
        В $ 3,3 $ рази.
    
\end{Solution}
\begin{Solution}{6.{23}}
        $ 23,6 $ ккд/м\tss{2}.
    
\end{Solution}
\begin{Solution}{6.{24}}
        $ P_R = 1$; $P_D = 9,05\% $.
    
\end{Solution}
\begin{Solution}{6.{25}}
        a) $ 1,63 $; б) $ \epsilon_\text{пр} = \arcsin(tg42^\circ37') = 66^\circ56'18'' $.
    
\end{Solution}
\begin{Solution}{6.{26}}
        $ \theta = 2(90 - \epsilon_\text{Б}) = 67^\circ22'48'' $.
    
\end{Solution}
\begin{Solution}{6.{27}}
        $ \theta = 60^\circ03'32'' $; $ \sigma_A = 59^\circ52'56'' $.
    
\end{Solution}
\begin{Solution}{6.{28}}
         а) $0,047$; б) $0,333$; в) $0,500$; г) $0,714$; д) $0,833 $.
    
\end{Solution}
\begin{Solution}{6.{29}}
        Світло повинно бути лінійно поляризованим у площині падіння та на гранях призми повинно бути $ R_{\parallel} = 0 $ . Звідси випливає, що показник заломлення скла призми повинен бути $ n = 1,732 $.
    
\end{Solution}
\begin{Solution}{6.{30}}
        $ \epsilon_B = 58^\circ46'54'' $; $ \frac{I_\text{П}}{I_\text{пр}} = 0,107 $; $ \phi = 35^\circ15'52'' $.
    
\end{Solution}
\begin{Solution}{6.{31}}
        а) Зменшиться у $ 1,5 $ рази; б) $ 69^\circ17'43'' $.
    
\end{Solution}
\begin{Solution}{6.{32}}
        $ I = I_0 \cos^2 \alpha \cos^2(\alpha_2 - \alpha_1) $; а) $ 66,22 $~Вт/м\tss{2}; б) $ 50,0 $~Вт/м\tss{2}; в) $ 46,65 $~Вт/м\tss{2}.
    
\end{Solution}
\begin{Solution}{6.{33}}
        а) $ \lambda_o = 355,2$~нм; $\lambda_e = 396,4 $~нм; $v_o = 1,809\cdot10^{10} $~см/с; $v_e = 2,019\cdot10^{10}$~см/с; б) $ \lambda_o = 382,5 $~нм; $\lambda_e = 380 $~нм; $v_o = 1,948\cdot10^{10} $~см/с; $v_e = 1,935\cdot10^{10}$~см/с;
    
\end{Solution}
\begin{Solution}{6.{34}}
        а) З обох призм виходить незвичайний промінь, з першої --- поляризований горизонтально, з другої --- вертикально; б) друга призма пропускає у $ 1,41 $ рази більше світла; в) $ \frac{1}{n_o} < \sin \alpha<\frac{1}{n_e} $; $ 37^\circ05'41''<\alpha<42^\circ17'42'' $.
    
\end{Solution}
\begin{Solution}{6.{35}}
        $ \phi = 2(n_o - n_e)tg\alpha = 5^\circ17' $.
    
\end{Solution}
\begin{Solution}{6.{36}}
        а) $ I = I_0(1 + 2\sin2\alpha\sin\delta)$; б) кварц --- додатній кристал ($ \delta > 0 $); при $ \alpha_1 = \frac{\pi}{4} \Rightarrow I_{\max} $, при $ \alpha_2 = \frac{3\pi}{4} \Rightarrow I_{\min}$.
    
\end{Solution}
\begin{Solution}{6.{37}}
        $ I = 2I_0(1 + \cos\delta) $; $ m = d_{\max} \frac{n_e - n_o}{\lambda} $; а) $ 10 $; б) $ 172 $.
    
\end{Solution}
\begin{Solution}{6.{38}}
        $ \approx 0,07  $~мм.
    
\end{Solution}
\begin{Solution}{6.{39}}
        а) $ F = \frac{\lambda a}{4k} = 31,25 $~Н; б) $ \frac{I_\perp}{I_\text{пр}} = 25\% $.
    
\end{Solution}
\begin{Solution}{6.{40}}
         а)  $ \frac{I_\perp}{I_\text{пр}} = 0 $; б) $\frac{I_\parallel}{I_\text{пр}} = 50\% $.
    
\end{Solution}
\begin{Solution}{6.{41}}
        $ (n_e - n_o)' = B\lambda E^2 = 0,12 \cdot 10^{-6} $; $ \delta = 2\pi BdE^2 = 2,85^\circ $.
    
\end{Solution}
\begin{Solution}{6.{42}}
        $ d = 133,6 $~мм.
    
\end{Solution}
\begin{Solution}{6.{43}}
        а) $ \delta = 2\pi cdH^2 = 3^\circ50'28'' $; б) $ \frac{I_H}{I_\text{пр}} = 49,\% $.
    
\end{Solution}
\begin{Solution}{6.{44}}
        а) $ d = 4,5 $~мм; б) $ d = 1,76 $~мм; в) $ d = 9 $~мм.
    
\end{Solution}
\begin{Solution}{6.{45}}
        а) $ d = 9 $~мм; б) $ d = 2,73 $~мм; в) $ d = 4,5 $~мм.
    
\end{Solution}
\begin{Solution}{6.{46}}
        а) $ \psi = VdH = 7^\circ14'42'' $; б) $ \frac{I_\perp}{I_\text{пр}} = 2\% $.
    
\end{Solution}
\begin{Solution}{6.{47}}
        а) $ V = \frac{\phi_1 + \phi_2}{2dH} = 3,246 \cdot 10^{-3}$~кут.хв./А; б) $ \psi = 45^\circ $; в) $ H = 2,772\cdot10^6 $~А/м.
    
\end{Solution}
\begin{Solution}{6.{48}}
        $ E_{\min} = \frac{1}{\sqrt{R\lambda B}} \approx 2,88 $~од.~СГСЕ $ \approx 900 $~В/см.
    
\end{Solution}
\begin{Solution}{6.{49}}
        $ \sqrt{2}/2 $.
    
\end{Solution}
\begin{Solution}{6.{50}}
        $ I = 3I_0 $.
    
\end{Solution}
\begin{Solution}{6.{51}}
        $ I = 5I_0 $.
    
\end{Solution}
\begin{Solution}{6.{52}}
        а) Інтенсивність світла збільшиться у $ 25 $ разів; 2) Інтенсивність світла збільшиться у $ 9 $ разів.
    
\end{Solution}
\begin{Solution}{6.{53}}
        $ I = 2I_0 $.
    
\end{Solution}
\begin{Solution}{6.{54}}
        $ I(\theta) = 3I_0 \left[1 + \frac{2}{3} \cos \left(\frac{4\pi d}{\lambda} \sin\theta\right)\right]  $, звідки $ V = \frac{2}{3} $; $ \Lambda = \frac{\lambda f}{2 d} = 2,5 \cdot 10^{-3} $~см.
    
\end{Solution}
\begin{Solution}{6.{55}}
        Повна різниця фаз світла, що приходить у максимум $ m $-го між двома взаємно
        перпендикулярними коливаннями
        \begin{equation*}
            \Delta \phi_m = \frac{\pi}{2}(m + 1) .
        \end{equation*}
        При $ m = 0 $, $\Delta \phi_0 = \frac{\pi}{2} $ --- кругова поляризація зі збереженням обертання світла, що падає;

        при $ m = \pm1 $, $\Delta \phi_1 = \pi $; $ \Delta \phi_{-1} = 0 $ ---  лінійно поляризоване світло;

        при $  m = \pm2 $ --- кругова поляризація з протилежним по відношенню до падаючого світла обертанням.

        У загальному випадку: у непарних максимумах світло лінійно поляризоване; при $  m = 0, \pm4, \pm8, \ldots $ кругова поляризація зі збереженням напряму обертання; при $ m = 2, \pm6, \pm10, \ldots $ кругова поляризація зі зворотнім обертанням.
    
\end{Solution}
\begin{Solution}{6.{56}}
        Зменшиться вдвічі, незалежно від поляризації світла, що падає.
    
\end{Solution}

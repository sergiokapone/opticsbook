% !TeX program = lualatex
% !TeX encoding = utf8
% !TeX spellcheck = uk_UA
% !TeX root =../OpticsProblems.tex

\clearpage
\section{Математичні формули та співвідношення}

\subsection{Тілесний кут}

Тілесний кут --- це відношення площі сфери, що перетинається конусом до квадрату її радіусу.

Елемент тілесного кута визначається як:
\begin{equation}\label{eq:solid_angle}
    d\Omega = \frac{dS}{r^2}
\end{equation}
де $dS$ --- елемент площі поверхні сфери радіуса $r$ (рис.~\ref{pis:solid_angle}).

%---------------------------------------------------------
\begin{figure}[h!]\centering
% 3D AXIS with spherical coordinates, dA
\tdplotsetmaincoords{60}{103}
\begin{tikzpicture}[scale=6,tdplot_main_coords]

    \colorlet{veccol}{themecolordark}
    \colorlet{projcol}{themecolordark}
    \tikzstyle{proj}=[projcol!80,line width=0.08] %very thin
    \tikzstyle{area}=[draw=veccol,fill=veccol!80,fill opacity=0.6]
    \tikzstyle{vector}=[-stealth,myblue,thick,line cap=round]
    \tikzstyle{unit vector}=[->,veccol,thick,line cap=round]
    \tikzstyle{dark unit vector}=[unit vector,veccol!70!black]
    \contourlength{1.3pt}

    % VARIABLE
    \def\rvec{1.0}
    \def\thetavec{35}
    \def\phivec{45}
    \def\dtheta{10}
    \def\dphi{16}
    \def\sphere#1#2#3{plot[domain=#1]({\rvec*sin(#2)*cos(#3)},{\rvec*sin(#2)*sin(#3)},{\rvec*cos(#2)})}
    \contourlength{0.8pt}

    % AXES
    \coordinate (O) at (0,0,0);
    \draw[thick,->] (0,0,0) -- (1.16*\rvec,0,0) node[left,below]{$x$};
    \draw[thick,->] (0,0,0) -- (0,1.1*\rvec,0) node[below,right]{$y$};
    \draw[thick,->] (0,0,0) -- (0,0,1.1*\rvec) node[above]{$z$};

    % COORDINATES
    \tdplotsetcoord{P}{\rvec}{\thetavec}{\phivec}
    \tdplotsetcoord{PB}{\rvec}{\thetavec+\dtheta}{\phivec}
    \tdplotsetcoord{PR}{\rvec}{\thetavec}{\phivec+\dphi}
    \tdplotsetcoord{PBR}{\rvec}{\thetavec+\dtheta}{\phivec+\dphi}

    % CONE
    \draw[veccol!20,very thin] (O)  -- (PBR);
    \draw[veccol!20,very thin] (O)  -- (PR);
    \draw[->,veccol] (O)  -- (P) node[below,left] {$\vb{r}$};
    \draw[veccol,very thin] (O)  -- (PB);

    % PROJECTIONS
    \draw[proj] %\thetavec+\dtheta
    plot[domain=0:90]({\rvec*sin(\x)*cos(\phivec)},{\rvec*sin(\x)*sin(\phivec)},{\rvec*cos(\x)}) coordinate (BL);
    \draw[proj]
    plot[domain=0:90]({\rvec*sin(\x)*cos(\phivec+\dphi)},{\rvec*sin(\x)*sin(\phivec+\dphi)},{\rvec*cos(\x)}) coordinate (BR);
    \draw[proj]
    plot[domain=0:90]({\rvec*cos(\x)},{\rvec*sin(\x)},0);
    \draw[proj] (O)  -- (BL); % PBxy
    \draw[proj] (O)  -- (BR); % PBRxy
    \draw[proj] (P)  -- (Pz) node[midway, above=1, sloped] {\contour{white}{$r\sin\theta$}};
    \draw[proj] (PR)  -- (Pz) ;
    %\draw[proj,projcol!15,dashed] (P) -- (Pxy);
    %\draw[proj,projcol!15,dashed] (PR) -- (PRxy);
    %\draw[proj,projcol!15,dashed] (PB) -- (PBxy);
    %\draw[proj,projcol!15,dashed] (PBR) -- (PBRxy);

    % AREA
    \draw[area]
    plot[domain=0:.99*\dphi]({\rvec*sin(\thetavec)*cos(\phivec+\x)},{\rvec*sin(\thetavec)*sin(\phivec+\x)},{\rvec*cos(\thetavec)}) --
    plot[domain=0:.99*\dtheta]({\rvec*sin(\thetavec+\x)*cos(\phivec+\dphi)},{\rvec*sin(\thetavec+\x)*sin(\phivec+\dphi)},{\rvec*cos(\thetavec+\x)}) --
    plot[domain=.99*\dphi:0]({\rvec*sin(\thetavec+\dtheta)*cos(\phivec+\x)},{\rvec*sin(\thetavec+\dtheta)*sin(\phivec+\x)},{\rvec*cos(\thetavec+\dtheta)}) --
    plot[domain=.99*\dtheta:0]({\rvec*sin(\thetavec+\x)*cos(\phivec)},{\rvec*sin(\thetavec+\x)*sin(\phivec)},{\rvec*cos(\thetavec+\x)}) --
    cycle;

    % MEASURES
    %\node[right=3,below right=-2] at (PB) {$r\sin\theta\dd{\phi}$};
    %\node[right=5,below right=-2] at (PR) {$r\dd{\theta}$};
    \draw[<->,
    proj,
    thin,
    insert node={
        \node[below, rotate=20]{\contour{white}{$r\sin\theta\dd{\phi}$}};
    } at 0.5
    ]
    plot[domain=0:\dphi]({\rvec*sin(\thetavec+1.11*\dtheta)*cos(\phivec+\x)},{\rvec*sin(\thetavec+1.11*\dtheta)*sin(\phivec+\x)},{\rvec*cos(\thetavec+1.11*\dtheta)});
    \draw[<->,proj,thin,
    insert node={
        \node[above, rotate=-50]{\contour{white}{$r\dd{\theta}$}};
    } at 0.5
    ]
    plot[domain=0:\dtheta]({\rvec*sin(\thetavec+\x)*cos(\phivec+1.15*\dphi)},{\rvec*sin(\thetavec+\x)*sin(\phivec+1.15*\dphi)},{\rvec*cos(\thetavec+\x)}) ;

    % ANGLES
    \tdplotdrawarc[->]{(O)}{0.35*\rvec}{0}{\phivec}
    {below}{$\phi$}
    \tdplotdrawarc[->]{(O)}{0.45*\rvec}{\phivec}{\phivec+\dphi}
    {anchor=145,inner sep=1}{\contour{white}{$\dd{\phi}$}}
    \tdplotsetthetaplanecoords{\phivec}
    \tdplotdrawarc[->,tdplot_rotated_coords]{(0,0,0)}{0.36*\rvec}{0}{\thetavec}
    {right,above right}{$\theta$}
    \tdplotdrawarc[->,tdplot_rotated_coords]{(0,0,0)}{0.54*\rvec}{\thetavec}{\thetavec+\dtheta}
    {left,above right}{\contour{white}{$\dd{\theta}$}}

\end{tikzpicture}
    \caption{Тілесний кут}
    \label{pis:solid_angle}
\end{figure}
%---------------------------------------------------------

Елемент тілесного кута в сферичній системі координат
\begin{equation}\label{eq:solid_angle_in_spherical}
    d\Omega = \sin\theta d\theta d\phi
\end{equation}
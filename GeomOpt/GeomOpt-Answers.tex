\protect \section *{\nameref *{GeomOpt}}
\begin{Solution}{1.{10}}
	а) $22,68$ мм; б) $13,65$ мм; в) $9,06$ мм.
\end{Solution}
\begin{Solution}{1.{11}}
	а) $6,36$ мм; б) $5,77$ мм
\end{Solution}
\begin{Solution}{1.{12}}
	$n_2\vect{r}_2 = n_1\vect{r}_0 - \left( n_1 (\vect{r}_0 \cdot \vect{N}) + \sqrt{n_2^2 - n_1^2 + n_1^2 (\vect{r}_0 \cdot \vect{N})^2} \right)\vect{N} $
\end{Solution}
\begin{Solution}{1.{14}}
	$16,3$~см.
\end{Solution}
\begin{Solution}{1.{15}}
	а) $11,4$ см; б) $9,13$ см.
\end{Solution}
\begin{Solution}{1.{16}}
	$41^\circ15'$.
\end{Solution}
\begin{Solution}{1.{17}}
	Не може.
\end{Solution}
\begin{Solution}{1.{18}}
	$h_{\min} = \frac{a\cos\phi}{\sin^2\phi \left( \frac{1}{\sqrt{n_1^2 - \sin^2\phi}}  - \frac{1}{\sqrt{n_2^2 - \sin^2\phi}} \right) }$.
\end{Solution}
\begin{Solution}{1.{19}}
	$6^\circ 57'4''$.
\end{Solution}
\begin{Solution}{1.{20}}
	$-4^\circ47'15''$.
\end{Solution}
\begin{Solution}{1.{21}}
	$82^\circ49'09''$.
\end{Solution}
\begin{Solution}{1.{22}}
	$8^\circ32'40''$.
\end{Solution}
\begin{Solution}{1.{23}}
	$-10^\circ08'27''$.
\end{Solution}
\begin{Solution}{1.{24}}
	$\sigma_{0_\text{ч}} = 30^\circ38'07''$; $\sigma_{0_\text{ф}} = 33^\circ27'09''$.
\end{Solution}
\begin{Solution}{1.{25}}
	$1^\circ12'$.
\end{Solution}
\begin{Solution}{1.{26}}
	$10$~см.
\end{Solution}
\begin{Solution}{1.{27}}
	$40$~см.
\end{Solution}
\begin{Solution}{1.{28}}
	$60$~см.
\end{Solution}
\begin{Solution}{1.{29}}
	а) $f = \frac{\beta L}{1 - \beta^2} = 10$~см; б) $f = \frac{\beta_1\beta_2 L}{\beta_2 - \beta_1} = 7.5$~см.
\end{Solution}
\begin{Solution}{1.{31}}
	$6$~м.
\end{Solution}
\begin{Solution}{1.{32}}
	$-20$ см; $3$ см.
\end{Solution}
\begin{Solution}{1.{38}}
	$15$~см.
\end{Solution}
\begin{Solution}{1.{39}}
	$48$~см.
\end{Solution}
\begin{Solution}{1.{40}}
	$2,08$~мм/с.
\end{Solution}
\begin{Solution}{1.{41}}
	$3,84$~см.
\end{Solution}
\begin{Solution}{1.{42}}
	$n = 2$.
\end{Solution}
\begin{Solution}{1.{43}}
	$-0,75$~дптр
\end{Solution}
\begin{Solution}{1.{44}}
	$n = 1,4$.
\end{Solution}
\begin{Solution}{1.{45}}
	$\frac{f_1}{f_2} = \frac{n}{n - 1}$.
\end{Solution}
\begin{Solution}{1.{46}}
	Фокусна відстань системи $f = \frac{f_1f_2}{f_1 + f_2 - l}$.  Відстань головних площин $H$ та $H'$ системи від першої та другої лінз відповідно дорівнюють: $|O_1H| = \frac{f_1l}{l - f_1 - f_2}$, $|O_1H'| = \frac{f_2l}{l - f_1 - f_2}$, відповідно.
\end{Solution}
\begin{Solution}{1.{48}}
	$-1,33$~дптр.
\end{Solution}
\begin{Solution}{1.{49}}
	$24$~дптр.
\end{Solution}
\begin{Solution}{1.{50}}
	$-8$~см.
\end{Solution}
\begin{Solution}{1.{51}}
	$-14$~дптр.
\end{Solution}
\begin{Solution}{1.{52}}
	$20$~см.
\end{Solution}
\begin{Solution}{1.{53}}
	$6$.
\end{Solution}
\begin{Solution}{1.{54}}
	$6$~мм.
\end{Solution}
\begin{Solution}{1.{55}}
	$100$.
\end{Solution}
\begin{Solution}{1.{56}}
	$\Gamma = 250$; $a = 10,5$~мм
\end{Solution}
\begin{Solution}{1.{57}}
	$2$~см.
\end{Solution}
\begin{Solution}{1.{58}}
	$80$.
\end{Solution}
\begin{Solution}{1.{59}}
	$7,5$~мкм.
\end{Solution}
\begin{Solution}{1.{60}}
        Може. Обернене зображення даватиме дво-опукла лінза  Пряме зображення даватиме опукло-увігнута лінза.  Товщина таких лінз визначається,  З урахуванням правила знаків, як
        \begin{equation*}
            d = \frac{n}{n - 1} (R_1 - R_2).
        \end{equation*}
    Збільшення такої лінзи (з урахуванням правила знаків):
    \begin{equation*}
        \Gamma = \frac{R_1}{R_2}.
    \end{equation*}
    Для опукло-впуклої лінзи, яка дає пряме зображення, має бути $|R_1| > |R_2|$.
    
\end{Solution}
\begin{Solution}{1.{61}}
	$12$
\end{Solution}
\begin{Solution}{1.{62}}
	1) до об’єктива на $1$~мм; 2) від об’єктива на $9$~мм.
\end{Solution}
\begin{Solution}{1.{63}}
	$30,7$~см.
\end{Solution}
\begin{Solution}{1.{64}}
	$h = v f'_\text{об} \frac{\tau}{\delta} = 2500$~м.
\end{Solution}
\begin{Solution}{1.{65}}
	\emph{Підказка}: Скористатись співвідношеннями \eqref{eq:Phi_of_thick_lens} або \eqref{eq:Phi_of_thick_lens2}.
	Закріпивши скельця за допомогою каучуку в трубчастому стволі і
	заповнивши простір між ними водою, можна отримати товсту водяну
	лінзу, яка слугуватиме зоровою трубою за умови телескопічності $\Phi = 0$,
	для чого відстань між лінзами має бути
	\begin{equation*}
		d = \frac{n}{n - 1}(R_1 - R_2) \approx \frac{4/3}{1/3}(40-16) = 96\ \text{см}.
	\end{equation*}
	Збільшення такої труби
	\begin{equation*}
		\Gamma = \frac{R_1}{R_2} = 2.5
	\end{equation*}
\end{Solution}
\begin{Solution}{1.{66}}
	\emph{Підказка}: Наповнивши пляшку водою і закоркувавши, так щоб не
	було бульбашки, через циліндричну частину пляшки, яку потрібно
	тримати паралельно до поверхні землі, сфокусувати сонячне проміння
	на листку паперу. Така циліндрична лінза дасть зображення у вигляді
	тонкої смужки сонячного світла на папері в напрямку прямих сонячних
	променів. За допомогою рулетки знайти радіус пляшки та фокусну
	відстань $f'$, яка відраховується від осі пляшки до сфокусованого
	зображення на папері.

	Показник заломлення знайдеться як:
	\begin{equation*}
		n = \frac{2f'}{2f' - R}.
	\end{equation*}
	Має бути
	\begin{equation*}
		n \approx \frac{4R}{4R - R} = 4/3.
	\end{equation*}
\end{Solution}
